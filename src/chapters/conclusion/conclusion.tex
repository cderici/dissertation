\chapter{Conclusion \& Final Thoughts}

    \begin{chapterpoint}
        Conclude the dissertation.
    \end{chapterpoint}

    \section{Related Work}

        \begin{mainpoint}
            There is research related to higher-order dynamic VMs.

            There is research related to tracing JIT VMs for languages.

            There is research related to optimizing VM performance in a meta-tracing context.

            There is research related to self-hosting.

            There is research related to managing abstraction levels.

            There is research related to space concerns \& heap allocated continuations.

        \end{mainpoint}

    \section{Research Contributions}
		\begin{mainpoint}
			\begin{itemize}
                \item This demonstrates with evidence that it is possible to have a working and complete self-hosting functional programming language on a meta-tracing JIT compiler.
                \item We revealed and analyzed a fundamental problem with self-hosting on meta-tracing JITs and argued it is solvable, proposing solution avenues clearly worth further study.  This is a significant contribution to PL research.
                \item We detailed a big problem in tracing JITs in general--you lose more when slow than you gain when fast, especially in branch-heavy code--and showed why solving it would be a major contribution.
                \item We built a solid tool (Pycket), useful for further research into both language runtimes and self-hosting.
                \item The research reframes performance analysis of tracing JITs: instead of only looking for hot loops to trace, identifying what to avoid tracing can unlock full potential.
                \item The research expands the way we think about runtimes, rather than an isolated separate part of a language implementation, a runtime can literally be a part of the language, and that the language can have an opinion about how the runtime looks like.
		    \end{itemize}
		\end{mainpoint}

		\begin{paragraph-here}
			A working system where you run full Racket (is a contribution).
		\end{paragraph-here}

    \section{Significance}
		\begin{mainpoint}
			% Why is this research cool/exciting/important?
			Paves the way and provides an example for future work on
			different ways of self-hosting languages.

			It's a step towards more intelligent communication between
			the language and the runtime.

			Having another Racket implementation symmetric to RacketCS
			enables further research on language runtimes for functional
			programming languages.

			It's also beneficial for furthering meta-tracing research.

            The research opens up new avenues of research\dots
		\end{mainpoint}

    \section{Future Work}
        \begin{mainpoint}
            There are a few future directions for this research:

            \begin{itemize}
                \item Solve all the issues with self-hosting on meta-tracing JIT, and realize
            the full potential of the approach.
                \item Investigate the language-runtime communication and how it can be improved.
                \item Survey the characteristics of different VMs (e.g. RacketCS vs Pycket)
                across a variety of computations.
                \item Build on the analysis we provide for the tracing performance of
            branch-heavy code, and make tracing JITs cool again.
            \end{itemize}
        \end{mainpoint}