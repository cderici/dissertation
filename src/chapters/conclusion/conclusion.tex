\chapter[\texorpdfstring{CONCLUSION \& FINAL THOUGHTS}
                          {8. Conclusion}]{CONCLUSION \& FINAL THOUGHTS}

    \label{chapter:conclusion}

    \begin{chaptersynopsis}[Chapter Synopsis - \emph{Chapter Content: 0\%}]
        Conclude the dissertation.

        \vspace{2em}

        Sections:
		\begin{itemize}
            \item Contributions \& Significance
			\item Related Work
			\item Future Work
		\end{itemize}
    \end{chaptersynopsis}

    \paragraph{} \textit{Thesis:}

    \textit{Efficient self-hosting of a full-scale functional language on a meta-tracing \gls{jit} compiler is achievable.}

    \begin{paragraph-here}% 1
        We detailed our reproducible research that proves our claim...
    \end{paragraph-here}

    \begin{paragraph-here}% 2
        In this chapter we provide some related work and conclude by listing concrete contributions we made towards proving our thesis. Finally we close with a discussion of broader significance and future directions for this work.
    \end{paragraph-here}

    \section[\texorpdfstring{Contributions \& Significance}{Significance}]{Contributions \& Significance}
        \begin{paragraph-here}% 3
            This study demonstrates with evidence that it is indeed possible to have a working full-scale self-hosting functional programming language on a meta-tracing JIT compiler.
        \end{paragraph-here}

        \begin{paragraph-here}% 4
            We built a useful research vehicle, namely Pycket, that can be utilized to further research into language run-times, self-hosting, meta-tracing and more. % pycket
        \end{paragraph-here}

        \begin{paragraph-here}% 5
            We provide for the first time a formalization for the operational semantics of linklets, a compilation unit that's critical in improving the communication between a compiler and a run-time, thereby improving portability of a programming language. % linklets
        \end{paragraph-here}

        \begin{paragraph-here}% 6
            We revealed and analyzed some performance problems fundamental to self-hosting on meta-tracing \glspl{jit} and argued it is solvable, proposing solution avenues clearly worth further study.  This is a significant contribution to PL research. % problems with self-hosting
        \end{paragraph-here}

        \begin{paragraph-here}% 7
            The research also reframes performance analysis of tracing \glspl{jit}: instead of only looking for hot loops to trace, identifying what to avoid tracing can unlock full potential.
        \end{paragraph-here}

        \begin{paragraph-here}% 8
            The research provides a novel way to think about computational models in language run-times. Instead of sticking to one, several different kinds of models can be combined to exploit strengths in different areas in implementing reduction semantics. % hybrid model
        \end{paragraph-here}

    \section{Related Work}
        \label{section:related-work}

        \begin{paragraph-here}% 9
             There is research related to higher-order dynamic VMs.
        \end{paragraph-here}

        \begin{paragraph-here}% 10
            There is research related to tracing JIT VMs for languages.
        \end{paragraph-here}

        \begin{paragraph-here}% 11
            There is research related to optimizing VM performance in a meta-tracing context.
        \end{paragraph-here}

        \begin{paragraph-here}% 12
            There is research related to self-hosting.
        \end{paragraph-here}

        \begin{paragraph-here}% 13
            There is research related to managing abstraction levels.

            Collapsing towers of interpreters? Futamura projections?
        \end{paragraph-here}

        \begin{paragraph-here}% 14
            There is research related to space concerns \& heap allocated continuations.
        \end{paragraph-here}

    \section[\texorpdfstring{Future Work}{Future Work}]{Future Work}

        \begin{paragraph-here}% 15
            There are a lot of dynamic features of programming languages, overhead of which can be eliminated by meta-tracing, such as eliminating gradual typing performance with using Pycket \cite{pycketmain2}. If all the issues of self-hosting on meta-tracing JITs are solved and the full potential of our approach is realized, then a featureful language like Racket and all languages that can be built on Racket can be made extremely fast with minimal effort.
        \end{paragraph-here}

        \begin{paragraph-here}% 16
            Having another Racket implementation symmetric to RacketCS enables further research on language runtimes for functional programming languages across a variety of different types of computations. Strenghts and weaknesses of each can be identified and the whole research about VMs for dynamic languages can be improved.
        \end{paragraph-here}

        \begin{paragraph-here}% 17
            This is an example of self-hosting with on a meta-tracing \gls{jit}, and it paves the way for future work on different ways of self-hosting languages.
        \end{paragraph-here}

        \begin{paragraph-here}% 18
            It opens interesting avenues for meta-tracing research, where we also touch one of the most classical problems with tracing \glspl{jit}; you lose more when slow than you gain when fast in evaluating branch-heavy computations.
        \end{paragraph-here}
