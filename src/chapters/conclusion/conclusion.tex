\chapter{Conclusion \& Final Thoughts}

    \begin{chapterpoint}
        The research expands the way we think about runtimes, rather than an isolated separate
        part of a language implementation, a runtime can literally be a part of the language, 
        and that the language can have an opinion about how the runtime looks like.

        The research reframes performance analysis of tracing JITs: instead of only looking for hot loops to trace, identifying what to avoid tracing can unlock full potential.

        We built a solid tool (Pycket), useful for further research into both language runtimes and self-hosting.

        The research opens up new avenues of research\dots

    \end{chapterpoint}

    \section{Related Work}

        \begin{mainpoint}
            There is research related to higher-order dynamic VMs.

            There is research related to tracing JIT VMs for languages.

            There is research related to optimizing VM performance in a meta-tracing context.

            There is research related to self-hosting.

            There is research related to managing abstraction levels.

            There is research related to space concerns \& heap allocated continuations.

        \end{mainpoint}

    \section{Future Work}
        \begin{mainpoint}
            There are a few future directions for this research:

            \begin{itemize}
                \item Solve all the issues with self-hosting on meta-tracing JIT, and realize
            the full potential of the approach.
                \item Investigate the language-runtime communication and how it can be improved.
                \item Survey the characteristics of different VMs (e.g. RacketCS vs Pycket) 
                across a variety of computations.
                \item Build on the analysis we provide for the tracing performance of
            branch-heavy code, and make tracing JITs cool again.
            \end{itemize}
        \end{mainpoint}