\chapter{Pycket as a Racket Host: Validation}

	\begin{chapterpoint}
		Bootstrapped Racket on Pycket works, and is correct and complete.
	\end{chapterpoint}

	\section{Pycket in Practice}

		\begin{mainpoint}
			Pycket as a Racket implementation works.
		\end{mainpoint}

	\section{Correctness by Construction}

		\begin{mainpoint}
			Pycket is developed into a full Racket implementation by a correctness preserving process. 
			
			(Started with the Old Pycket, added untouched Racket code on top of it.)
		\end{mainpoint}

		\subsection{Ensuring Correctness of the Linklet Layer}
			\begin{mainpoint}
				We have 100\% coverage on the operational semantics as unit tests.
				
				Additionally, we symmetrically used the same set of tests with the redex model that realize the operational semantics.
			\end{mainpoint}

		\subsection{Correctness of Pycket as a Racket Runtime}
			\begin{mainpoint}
				We add untouched Racket code on top of Old Pycket to expand functionality.

				Correctness is preserved.
			\end{mainpoint}

		\subsection{Overall Correctness of the System}

			\begin{mainpoint}
				... is ensured by the Old and New Unit and end-to-end/integration tests.
			\end{mainpoint}

	\section{Completeness under Self-Hosting}

		\begin{mainpoint}
			Pycket can evaluate any \#lang that Racket evaluates.
		\end{mainpoint}

		\subsection{From \#\%kernel to \#lang racket/base}
			\begin{mainpoint}
				Racket is a programming language programming language.

				As long as you have the expander to reduce the given language to \#\%kernel, you're golden.
			\end{mainpoint}
		
		\subsection{Beyond \#lang racket/base}
			\begin{mainpoint}
				Being able to implement \#lang racket/base is a good springboard to implement larger languages.
			\end{mainpoint}