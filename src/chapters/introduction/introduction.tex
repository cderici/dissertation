\chapter[\texorpdfstring{INTRODUCTION}
                          {1. Introduction}]{INTRODUCTION}
    \label{chapter:introduction}

    \begin{chaptersynopsis}[Chapter Synopsis - \emph{Content Complete}]

       Thesis:

        \textit{Efficient self-hosting of full-scale functional languages on meta-tracing \gls{jit} compilers is achievable.}

        \vspace{2em}

        Sections:
		\begin{itemize}
			\item Motivation \& Context
			\item Dissertation Structure \& Contributions
		\end{itemize}
    \end{chaptersynopsis}

    \section{Thesis}

    \textit{Efficient self-hosting of a full-scale functional language on a meta-tracing \gls{jit} compiler is achievable.}


    \section[\texorpdfstring{Motivation \& Context}{Context}]{Motivation \& Context}

        \paragraph{}% 1
            One significant milestone in the evolution of a programming language implementation is the capability to compile or interpret its own source code. This process, historically known as “meta-circular evaluation” \cite{sicpbook}, is now commonly referred to as bootstrapping, or as we will call it throughout this dissertation, self-hosting. Self-hosting requires the language to possess certain essential features, such as expressive completeness and a robust run-time system. Therefore, the ability to self-host indicates a language's semantic maturity, broadly considered as a combination of expressive power and practical real-world applicability.

        \paragraph{}% 2
            Self-hosting dynamic programming languages inherently involves challenging performance trade-offs, even in specialized research compilers such as Tachyon for JavaScript \cite{self-hosted-tachyon}, and these challenges become increasingly complex for fully developed, real-world language run-times. Ideally, we would like a canonical method to implement efficient run-times for dynamic languages while retaining semantic maturity. This approach would enhance language portability, enabling practical benefits such as leveraging the unique capabilities of different virtual machines (VMs) for various computational tasks without the need to re-implement the language repeatedly. Such redundant re-implementations are not only error-prone but also complicate performance comparisons across different VMs.

        \paragraph{}% 3
            Meta-tracing has proven to be an effective implementation technique for dynamic programming languages, as demonstrated by Bolz in his foundational work \cite{bolzPhDThesis}. Unlike traditional meta-circular approaches, meta-tracing enables the semantics of a language to be specified through an interpreter, automatically generating a virtual machine (via C code) equipped with a specialized tracing \gls{jit} compiler. This method has been successfully applied in PyPy, a Python implementation that significantly outperforms standard CPython in runtime performance \cite{pypy-main}. Thus, meta-tracing represents a promising strategy for achieving both ease of language implementation and efficiency at the same time.

        \paragraph{}% 4
            Given the practical advantages of self-hosting and the demonstrated strengths of meta-tracing, we hypothesize that efficient self-hosting of a full-scale functional language on a meta-tracing \gls{jit} compiler is achievable. This dissertation explores this open research question by investigating whether a meta-tracing run-time system can indeed serve as an efficient virtual machine for a self-hosting, general-purpose, dynamic functional programming language.

        \paragraph{}% 5
            We choose Racket as our main language for demonstrating this hypothesis. Racket is a particularly suitable candidate because it is reasonably well-adopted, explicitly designed to facilitate programming language research, and has recently enhanced its portability by migrating its core run-time to Chez Scheme. This transition has significantly simplified targeting different virtual machines, making Racket an ideal language for exploring efficient self-hosting on a meta-tracing \gls{jit} compiler \cite{icfp2019}.

        \paragraph{}% 6
            Additionally, Racket has an existing, although rudimentary, meta-tracing interpreter named Pycket, developed as a research project using the RPython framework—the same meta-tracing framework on which PyPy was built \cite{bolzPhDThesis,bolzMetatracingMakesFast2014}. Pycket has already demonstrated notable performance improvements and has been shown to eliminate 90\% of the performance overhead associated with gradual typing \cite{pycketmain,pycketmain2}. These characteristics position Pycket as an excellent candidate for this study, providing a solid starting point for validating our hypothesis.

        \paragraph{}% 7
            To substantiate our hypothesis, we first enhance Pycket from a simple interpreter to a full-scale virtual machine capable of hosting Racket. Leveraging Racket's improved self-hosting capabilities, we demonstrate concretely that an efficient, self-hosting implementation of a full-scale functional language on a meta-tracing \gls{jit} compiler is achievable. This transformation provides clear evidence in support of our primary thesis statement.

        \paragraph{}% 8
            During this process, we identify and analyze several performance issues fundamental to self-hosting languages using meta-tracing compilers. While these issues pose challenges, we provide evidence that such performance limitations can be effectively addressed and mitigated, reinforcing our claim that efficient self-hosting on meta-tracing JIT compilers is feasible.

    \section[\texorpdfstring{Dissertation Structure \& Contributions}{Structure \& Contributions}]{Dissertation Structure \& Contributions}

        \paragraph{}% 9
            This dissertation is structured as follows. We start by examining meta-tracing \gls{jit} compilers in \chapterRef{chapter:rpython}, providing a primer on Pycket and its underlying RPython framework. We then analyze the concept of portability in programming languages in \chapterRef{chapter:linklets}, giving a detailed explanation of Racket's key innovation that significantly improved its portability.

        \paragraph{}% 10
            In \chapterRef{chapter:pycket}, we describe our methodology and detail the steps taken to achieve a full-scale, self-hosting implementation of Racket on Pycket, addressing the first part of our thesis statement. We validate the correctness and completeness of our resulting system in \chapterRef{chapter:validation}, confirming that Pycket indeed functions as a complete Racket implementation.

        \paragraph{}% 11
            Next, we address performance considerations in \chapterRef{chapter:problem}. We present a thorough evaluation of self-hosting on meta-tracing \gls{jit} compilers, identifying and analyzing critical performance bottlenecks that inherently arise in this approach. In \chapterRef{chapter:solution}, we explore various strategies to overcome these performance challenges. Through concrete evidence, we demonstrate that despite these fundamental limitations, self-hosting on a meta-tracing compiler can be efficiently realized.

        \paragraph{}% 12
            Finally, we conclude with a discussion of related work and broader implications of our findings in \chapterRef{chapter:conclusion}.

        \paragraph{}% 13
            The primary technical contributions of this dissertation are:
            \begin{itemize}
                \item Pycket, the first self-hosting implementation of a full-scale functional programming language on a meta-tracing \gls{jit} compiler, serving as a complete run-time for Racket.
                \item The first formalization of linklet semantics, providing an operational semantics to clearly reason about their behavior.
                \item Rudimentary operational semantics for a hybrid computational model (\racketcode{CEK} \& Stackful) that utilizes both stack and the heap in an intelligent way for optimized memory use.
                \item Identification and thorough analysis of performance issues fundamental to self-hosting on meta-tracing \gls{jit} compilers.
            \end{itemize}

% \begin{todo}
% Paragraph still missing - decide whether to keep!
% \end{todo}



% This is a figure
% \begin{figure}
% 	\includegraphics[width=\textwidth]{\figPath{introduction}/exampleFigure.png}
% 	\caption{This is an example Figure.}
% 	\label{Figure in Chapter 1}
% \end{figure}

% Lorem ipsum dolor sit amet, consectetur adipiscing elit, sed do eiusmod tempor incididunt ut labore et dolore magna aliqua. Ut enim ad minim veniam, quis nostrud exercitation ullamco laboris nisi ut aliquip ex ea commodo consequat \cite{ref1}. Duis aute irure dolor in reprehenderit in voluptate velit esse cillum dolore eu fugiat nulla pariatur \cite{ref2}. Excepteur sint occaecat cupidatat non proident, sunt in culpa qui officia deserunt mollit anim id est laborum \cite{ref3}.

% Lorem ipsum dolor sit amet, consectetur adipiscing elit, sed do eiusmod tempor incididunt ut labore et dolore magna aliqua. Ut enim ad minim veniam, quis nostrud exercitation ullamco laboris nisi ut aliquip ex ea commodo consequat \cite{ref1}. Duis aute irure dolor in reprehenderit in voluptate velit esse cillum dolore eu fugiat nulla pariatur \cite{ref2}. Excepteur sint occaecat cupidatat non proident, sunt in culpa qui officia deserunt mollit anim id est laborum.

% Lorem ipsum dolor sit amet, consectetur adipiscing elit, sed do eiusmod tempor incididunt ut labore et dolore magna aliqua. Ut enim ad minim veniam, quis nostrud exercitation ullamco laboris nisi ut aliquip ex ea commodo consequat \cite{ref1}. Duis aute irure dolor in reprehenderit in voluptate velit esse cillum dolore eu fugiat nulla pariatur \cite{ref2}. Excepteur sint occaecat cupidatat non proident, sunt in culpa qui officia deserunt mollit anim id est laborum.

% Lorem ipsum dolor sit amet, consectetur adipiscing elit, sed do eiusmod tempor incididunt ut labore et dolore magna aliqua. Ut enim ad minim veniam, quis nostrud exercitation ullamco laboris nisi ut aliquip ex ea commodo consequat \cite{ref1}. Duis aute irure dolor in reprehenderit in voluptate velit esse cillum dolore eu fugiat nulla pariatur \cite{ref2}. Excepteur sint occaecat cupidatat non proident, sunt in culpa qui officia deserunt mollit anim id est laborum.

% Lorem ipsum dolor sit amet, consectetur adipiscing elit, sed do eiusmod tempor incididunt ut labore et dolore magna aliqua. Ut enim ad minim veniam, quis nostrud exercitation ullamco laboris nisi ut aliquip ex ea commodo consequat \cite{ref1}. Duis aute irure dolor in reprehenderit in voluptate velit esse cillum dolore eu fugiat nulla pariatur \cite{ref2}. Excepteur sint occaecat cupidatat non proident, sunt in culpa qui officia deserunt mollit anim id est laborum.

% Lorem ipsum dolor sit amet, consectetur adipiscing elit, sed do eiusmod tempor incididunt ut labore et dolore magna aliqua. Ut enim ad minim veniam, quis nostrud exercitation ullamco laboris nisi ut aliquip ex ea commodo consequat \cite{ref1}. Duis aute irure dolor in reprehenderit in voluptate velit esse cillum dolore eu fugiat nulla pariatur \cite{ref2}. Excepteur sint occaecat cupidatat non proident, sunt in culpa qui officia deserunt mollit anim id est laborum.

% This is a table

% \begin{table}
% 	\caption{This is an example Table.}
% 	\begin{center}
% 		\begin{tabular}{ccc}
% 			x & f(x) & g(x) \\
% 			\hline
% 			1 & 6    & 4    \\
% 			2 & 6    & 3    \\
% 			3 & 6    & 2    \\
% 			4 & 6    & 2    \\
% 			\label{Table in Chapter 1}
% 		\end{tabular}
% 	\end{center}
% \end{table}
