\chapter[\texorpdfstring{LANGUAGE-POWERED RUNTIMES}
                          {Self-hosting Pycket}]{LANGUAGE-POWERED RUNTIMES}
	\label{chapter:pycket}

	\begin{chaptersynopsis}
		Languages can influence the capabilities of the runtime they are implemented on.

		Self-hosting of Racket is realized on Pycket.

	\end{chaptersynopsis}

	\begin{paragraph-here}
		Pycket is a suitable host and environment for proving the self-hosting hypothesis.
	\end{paragraph-here}

	\section{Concrete Implementations of Linklets}

		\begin{mainpoint}
			Runtimes can represent linklets however they want.

			Chez Scheme represents them as functions, here's how Pycket represents and implements them.
		\end{mainpoint}

	\section{Enhancing Runtime with Bootstrapping Linklets}
		\begin{mainpoint}
			Importing functionality straight from the language in the form of linklets allows the language to shape the runtime.
		\end{mainpoint}
		\subsection{Interfacing with the Compiler: Expander Linklet}

		\subsection{Interfacing with the Host Environment: IO \& Thread Linklets}
			\begin{todo}[TODO]
				rktio
			\end{todo}
			\begin{todo}[TODO]
				engines
			\end{todo}
		\subsection{Language Utilities: Fasl \& Regexp Linklets}

	\section{Growing Pycket into full Racket}

		\begin{mainpoint}
            How it's all tied up together to demonstrate the self-hosting hypothesis.

			\textit{Self-hosting of full-scale functional languages on meta-tracing just-in-time (JIT) compilers is achievable.}
        \end{mainpoint}

        \begin{todo}
			A rudimentary interpreter can be bootstrapped into a full language implementation by importing linklets into the runtime.
		\end{todo}

		\begin{todo}[Import]
			Section 3.3 Making Pycket an Independent Self-Hosting Racket of the proposal document is relevant for this section.
		\end{todo}