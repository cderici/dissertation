\chapter[\texorpdfstring{FROM RUDIMENTARY INTERPRETER TO FULL LANGUAGE IMPLEMENTATION}
                          {4. Pycket as Full Racket}]{FROM RUDIMENTARY INTERPRETER TO FULL LANGUAGE IMPLEMENTATION}
	\label{chapter:pycket}

	\begin{chaptersynopsis}[Chapter Synopsis]
		\footnotesize

        Languages can influence the capabilities of the run-time they are implemented on.

		Concrete realization of self-hosting of Racket on Pycket.

        \vspace{2em}

        Sections:
		\begin{itemize}
			\item Implementing Linklets on Pycket

			This is how Pycket is enhanced to interface with linklets. Representation choices, etc.

			\item Enhancing Run-time with Bootstrapping Linklets

			Importing functionality straight from the language in the form of linklets allows the language to shape the behavior of the run-time.

			\item From Rudimentary Interpreter to Full Language Implementation

			Pycket becomes a full working Racket, proving the self-hosting hypothesis.

		\end{itemize}
    \end{chaptersynopsis}

	\paragraph{}% 1
		Languages can influence and enhance the capabilities of the run-time environments or \glspl{vm} they are implemented on. This influence is evident in mature language implementations such as Smalltalk's Squeak, Erlang's OTP on the BEAM \gls{vm}, and Common Lisp's CLOS Meta-Object Protocol (MOP). For example, Smalltalk allows core image manipulation within Smalltalk itself, extending \gls{vm} semantics dynamically \cite{squeak_smalltalk_vm}. Erlang uses its OTP libraries written in Erlang to dynamically control crucial runtime semantics like process scheduling and fault tolerance \cite{erlang_otp_hpc}. Similarly, Common Lisp employs the CLOS MOP to redefine fundamental object system semantics at runtime without altering the native runtime \cite{clos_overview_mop}.

	\paragraph{}% 2
		While the extent of language influence on the runtime varies considerably, Racket's linklets stand out by enabling the language to self-host itself. Linklets can carry crucial Racket subsystems, including the macro expander, the module system, and essential runtime components, to be expressed and distributed as language-level constructs that are supported by a thin primitive layer by the \gls{vm}. This mechanism significantly reduces dependency on the underlying runtime, thereby simplifying portability across different \glspl{vm}.

	\paragraph{}% 3
		Recall Pycket from \secref{section:pycket-primer}, initially introduced as a rudimentary interpreter for Racket, lacking capability required to independently execute real-world Racket programs. Pycket's initial design relied on the existing Racket executable to read and expand Racket modules before evaluation. In contrast, a \emph{full language implementation} refers to an environment capable of independently performing all necessary tasks required to load and execute programs written in that language. Such implementation can read source code, expand macros (load languages), evaluate modules, and provide the full spectrum of runtime facilities, such as a \gls{repl}, error handling, I/O operations, and more, without external assistance.

	\paragraph{}% 5
		Building upon the concepts introduced in \chapterRef{chapter:linklets}, this chapter presents concrete implementation details that transition Pycket from a rudimentary interpreter into a fully operational Racket implementation. By incorporating linklets, Pycket gains the ability to independently load, expand, and evaluate Racket modules, thus substantiating the self-hosting hypothesis outlined earlier. The following sections elaborate on the representation and implementation choices involved in this transition, highlighting runtime enhancements.

	\section[\texorpdfstring{Implementing Linklets on Pycket}{Linklets on Pycket}]{Implementing Linklets on Pycket}

		\paragraph{}% 6
			The first step in establishing an interface with linklets is to define their internal representation. Recall that the linklet design itself does not mandate a particular representation, leaving the decision to the runtime implementer. Chez Scheme, for example, represents linklets as first-class functions. Pycket, in contrast, represents both linklets and linklet instances as first-class Racket values using the custom \pycketcode{W_Linklet} and \pycketcode{W_LinkletInstance} classes, both derived from \pycketcode{W_Object} (the parent abstract class representing Racket values). \figref{fig:pycket-linklet-representation} lists the class definitions of linklets and linklet instances in Pycket. This choice facilitates seamless integration and leverages Pycket’s existing object model.

		\inputFigure{pycket}{pycket-linklet-representation-figure}

		\paragraph{}% 7
			For linklet variables, Pycket employs a specialized \gls{ast} node named \pycketcode{LinkletVar}, which evaluates to heap-allocated cells (\pycketcode{W\_Cell} objects) rather than separate dedicated runtime values. This approach allows the trace optimizer to perform effective optimizations on the top-level environment. Using cells for linklet variables also makes it possible to leverage specialized strategies implemented for cells, such as type specialization and inline caching, thus enhancing runtime efficiency. Consequently, these representation choices significantly influence the design and implementation of \emph{compile-linklet} and \emph{instantiate-linklet}, as these functions directly manifest the runtime’s internal representation for linklets.

		\paragraph{}% 8
			The \emph{compile-linklet} function transforms a linklet expressed as an s-expression into a runnable linklet object, following the formal semantics described in \secref{section:linklet-formal-semantics}. It begins by analyzing the s-expression through multiple passes to identify identifiers defined or mutated within the linklet body. Next, it processes the identifiers for imports and exports, creating runtime objects (\emph{Import} and \emph{Export}) and generating (via \emph{gensym}) fresh identifiers for internal use as needed. Finally, it recursively converts the linklet body into an RPython \gls{ast}, encapsulating all this information within a \pycketcode{W_Linklet} object ready for instantiation.

		\paragraph{}% 9
			The first phase of linklet compilation involves handling imports and exports. Each import specification, kept in a linklet as \pycketcode{[Import ...]} as seen in \figref{fig:pycket-linklet-representation}, references a specific imported instance, with each \pycketcode{Import} object corresponding to a variable provided by that instance. If the linklet specifies distinct names for external (provided by the imported instance) and internal (used within the linklet body) references, these mappings are recorded explicitly within an \pycketcode{Import} object. Likewise, exports are treated similarly; each exported variable may have separate external and internal identifiers, the former visible to importing linklets and the latter used internally within the body. This explicit mapping simplifies variable management and prevents naming conflicts across linklet boundaries.

		\paragraph{}% 10
			After imports and exports are handled, the compilation proceeds with processing the linklet body. Initially, it identifies all targets of \pycketcode{set!} expressions and variables introduced by \pycketcode{define-values} forms. Subsequently, each form within the body is individually compiled from s-expression to Pycket \gls{ast}. Most compilation rules follow straightforward translations, with the notable exceptions involving linklet-variable-specific forms—such as \pycketcode{variable-ref}, \pycketcode{variable-set!}, and related variations—rules of which are detailed in \tableRef{table:linklet-compilation-rules} in \chapterRef{chapter:linklets}. These forms explicitly handle interactions with linklet variables, ensuring correct runtime semantics during instantiation.

		\paragraph{}% 12
			For each top-level defined identifier that's also exported, an additional \pycketcode{variable-set!} form is inserted by the compiler to ensure the correct runtime assignment to the corresponding linklet variable. When an exported identifier appears as a target of a \pycketcode{set!}, the compiler translates the original form into a \pycketcode{variable-set!/check-undefined}, thereby ensuring proper handling of potentially uninitialized variables at runtime. References to exported identifiers that are either undefined at the top level or targeted by a \pycketcode{set!} are explicitly compiled into \pycketcode{variable-ref} forms, enabling dynamic resolution via target instance during instantiation. Finally, references to imported variables are transformed into \pycketcode{variable-ref/no-check} forms. These forms omit runtime existence checks since instantiation will already fail if the variable cannot be imported successfully. Once the entire linklet s-expression has been processed, the \emph{compile-linklet} function outputs a \pycketcode{W_Linklet} object containing all essential runtime information. Specifically, this object includes the prepared \emph{Import} and \emph{Export} mappings, as well as the compiled Pycket \gls{ast} representing the linklet body.

		\paragraph{}% 14
			On the other hand, \emph{instantiate-linklet} performs the actual evaluation of a compiled linklet, implementing the semantics described by the reduction relation in \figref{fig:linklets-reduction-relation}. Initially, it processes the \emph{Import} and \emph{Export} objects, collecting and creating linklet variables accordingly. If the number of instances provided for imports is insufficient, or if any import variable isn't exported by the corresponding input instance, instantiation halts immediately with an error. Next, the function ensures a target instance, either creating a new one during regular instantiation or using an explicitly provided instance during targeted instantiation. Finally, the instantiation proceeds by evaluating the body expressions using the CEK interpreter loop, loading specialized continuation frames (such as \pycketcode{instantiate_def_cont} and \pycketcode{instantiate_val_cont}) that effectively implement the transitive closure of the reduction relation $\longrightarrow_{\beta_p}$. Depending on the instantiation mode, it returns (by applying its own continuation to) either a value or a linklet instance (\racketcode{W_LinkletInstance}).

		\paragraph{}% 16
			Having implemented both \emph{compile-linklet} and \emph{instantiate-linklet}, Pycket gains the capability to independently execute any code expressed in Racket's core language (\pycketcode{#\%kernel}) provided in the form of linklet s-expressions. Together with the internal representations and runtime mechanisms for linklets detailed earlier, these functions establish the foundation that allows Pycket to import and execute Racket’s own runtime extensions, referred to as the bootstrapping linklets. These bootstrapping linklets carry implementations of substantial high-level Racket subsystems—including the macro expander, module system, and other critical runtime components—thereby enabling self-hosting on any runtime environment capable of loading and instantiating linklets. The next section explores the integration and utilization of these bootstrapping linklets within Pycket in detail.

	\section[\texorpdfstring{Enhancing Run-time with Bootstrapping Linklets}{Bootstrapping Linklets}]{Enhancing Run-time with Bootstrapping Linklets}
		% \begin{mainpoint}
		% 	Importing functionality straight from the language in the form of linklets allows the language to shape the behavior of the run-time.
		% \end{mainpoint}

		\paragraph{}% 1
			Bootstrapping linklets, as introduced in \chapterRef{chapter:linklets}, provide a powerful mechanism for encapsulating major Racket subsystems, such as the macro expander, module system, and runtime components, into independently loadable units. By applying the Racket macro expander to any given module (including itself), one can generate a collection of linklets that may then be flattened and merged into a single, standalone linklet s-expression. Using this technique, Racket exposes essential functionalities as separate bootstrapping linklets, notably the expander, IO, threads, and additional utility linklets like fasl and regexp, each of which will be discussed in subsequent subsections.

		\paragraph{}% 3
			With the capabilities provided by \racketcode{compile-linklet} and \racketcode{instantiate-linklet}, Pycket significantly transforms its startup process. Previously, Pycket depended on external tools for reading and expanding modules, as discussed in \secref{section:linklet-motivation}. Now, Pycket is able to independently load the bootstrapping linklets directly into its runtime during initialization, eliminating this dependency and enabling a fully self-contained bootstrapping phase.

		\inputFigure{pycket}{figure-loading-linklet-at-boot}

		\paragraph{}% 4
			\figref{fig:pycket-linklet-loading} illustrates Pycket's process of loading a linklet at startup. Pycket begins by reading the linklet s-expression, then compiles it into a \pycketcode{W_Linklet} object using \racketcode{compile-linklet}. Subsequently, it instantiates this compiled linklet via \racketcode{instantiate-linklet} to produce a linklet instance, which serves as a container holding all variables defined and exported by the linklet. These variables typically represent callable closures and other valid Racket values. Finally, Pycket exposes these variables in its global primitive environment, thereby making Racket-defined values—and more significantly, functions—directly callable from Pycket in a manner indistinguishable from its own primitives implemented in RPython.

		\paragraph{}% 5
			Applying this loading mechanism, Pycket sequentially imports and integrates the bootstrapping linklets in the following order: \emph{regexp}, \emph{thread}, \emph{io}, \emph{fasl}, and \emph{expander}. Each linklet enhances Pycket's runtime environment with specific Racket functionalities, exposing powerful primitives such as \racketcode{read}, \racketcode{write}, \racketcode{expand}, and \racketcode{namespace-require}. \tableRef{table:bootstrapping-linklets} summarizes the size of each linklet and the number of primitives each exposes. Notably, the expander linklet exports fewer primitives than the IO linklet despite having nearly three times as many lines of code.

		\inputFigure{pycket}{table-bootstrapping-linklets-info}

		\paragraph{}% 6
			In addition to standard bootstrapping linklets, Pycket loads a specialized linklet named \pycketcode{pycket_boot}, designed specifically for Pycket's internal use. Previously, Pycket manually invoked Racket’s \racketcode{dynamic-require} primitive from the global primitive environment to initialize the runtime, including loading core languages such as \racketcode{racket/base}. With the availability of the linklet infrastructure, all such startup procedures—previously implemented directly in RPython—can now be written directly in Racket, encapsulated within the \pycketcode{pycket_boot} linklet. This approach significantly simplifies Pycket’s frontend initialization code, integrating setup procedures and configuration parameters like \racketcode{current-library-collection-links}, \racketcode{read-accept-compiled}, and \racketcode{use-compiled-file-paths} entirely at the Racket level.

		\inputFigure{pycket}{figure-new-pycket-frontend-compare-old}

		\paragraph{}% 7
			The integration of the bootstrapping and custom linklets fundamentally transforms Pycket's frontend architecture, as depicted in \figref{fig:pycket-frontend-new-vs-old}. Whereas previously Pycket had to invoke the Racket executable externally to expand Racket modules before evaluation, it now applies Racket's macro expander internally to a given module and evaluates the resulting fully-expanded form-all directly within its CEK core. This architectural shift places all phases of module evaluation—reading, expansion, and execution—fully within Pycket’s runtime, thereby providing self-containment and independence. However, embedding the entire expansion phase internally has substantial implications for runtime performance, a topic examined in depth in \chapterRef{chapter:problem}.

		\paragraph{}% 8
			The bootstrapping linklets provided by Racket depend on primitives supplied by the host runtime—in this case, Pycket. When compiling these linklets, Pycket generates \gls{ast} nodes for each primitive reference. To ensure successful compilation, these primitives must be resolvable within Pycket's global primitive environment. While Pycket already implements a many Racket primitives in RPython—as detailed in \chapterRef{chapter:rpython}—some linklets, notably \emph{thread} and \emph{io}, require additional sub-systems such as engines and a \gls{ffi} layer, which will be discussed further in subsequent sections.

		\inputFigure{pycket}{figure-racket-base-loading-no-compiled}

		\paragraph{}% 9
			Loading bootstrapping linklets into the runtime environment serves as the central mechanism enabling self-hosting. Pycket can read and expand any Racket module by leveraging Racket's own module and macro systems, provided by the expander linklet. For example, loading a language such as \racketcode{#lang racket/base} involves the macro expander and module system generating individual linklets for each required module in its dependency tree. As illustrated in \figref{fig:racket-base-loading-no-compiled}, these modules are compiled and instantiated separately by Pycket. This mechanism is utilized in the next section, where we describe the use of pre-compiled Racket modules to achieve significantly faster startup performance.

		\subsection{Interfacing with the Compiler: Expander Linklet}

			\paragraph{}% 10
				As described in \chapterRef{chapter:linklets}, the expander linklet is generated offline by running the expander on itself, resulting in a serialized s-expression. Pycket reads this serialized s-expression, compiles it into a linklet object using \racketcode{compile-linklet}, and then instantiates it using \racketcode{instantiate-linklet}. Once instantiated, Pycket incorporates all exported functions from the expander—such as \racketcode{read}, \racketcode{expand}, and \racketcode{eval}—into its runtime environment, allowing direct invocation of these functions as native primitives.

			\begin{center}
				\begin{minipage}{0.7\textwidth}
					\begin{lstlisting}[style=inline-python,frame=lines,numbers=none]
	expander_linklet_obj = compile_linklet([expander_linklet_sexp, ...])
	expander_linklet_instance = instantiate_linklet([expander_linklet_obj, ...])
	expander_linklet_instance.expose_vars_to_prim_env()\end{lstlisting}
				\end{minipage}
			\end{center}

			\paragraph{}% 101
				Having direct access to the expander's runtime functions enables Pycket to implement language-level operations such as a top-level \gls{repl} through Racket's own primitives. A minimal implementation of a \gls{repl} can be realized simply by invoking Racket's \racketcode{read}, \racketcode{expand}, and \racketcode{eval} functions within Pycket, as illustrated in the small inline example below. Furthermore, since the expander linklet provides the complete implementations of Racket's macro and module systems, Pycket can directly execute programs written in languages layered on top of the Racket core, such as \racketcode{racket/base}. As a result, instead of implementing a custom \gls{repl}, Pycket can directly apply \racketcode{dynamic-require} to Racket's standard \racketcode{racket/repl} module. This yields a REPL functionally identical to Racket’s own, executed within Pycket’s meta-tracing JIT environment.

			\begin{center}
				\begin{minipage}{0.6\textwidth}
					\begin{lstlisting}[style=inline-python,frame=lines,numbers=none]
				while True:
					r_exp = read.call_racket([repl.readline(), ...])
					r_expanded = expand.call_racket([r_exp, ...])
					result = eval.call_racket([r_expanded, ...])
					print(result)\end{lstlisting}
				\end{minipage}
			\end{center}

			\paragraph{}% 11
				Programs executed through the bootstrapping linklets, including the linklets themselves, often require additional run-time support beyond basic primitives. For instance, running Racket's REPL necessitates implementing delimited continuations at the run-time level. Similarly, handling exceptions originating from Racket code demands robust Racket-level exception handling within Pycket. Pycket's existing implementation provides basic exception handling by translating Racket-level exceptions (implemented as Racket structs) into RPython exceptions. However, fully supporting Racket-level exceptions requires Pycket to handle the installation of exception handlers via continuation marks and dynamically propagate Racket-level exceptions to the appropriate (Racket level) handlers.

			\paragraph{}% 12
				Among the critical functions exposed by the expander linklet are \racketcode{namespace-require}, \racketcode{read}, \racketcode{expand}, and \racketcode{eval}, each of which embodies substantial and complex functionality. \racketcode{eval} functions effectively as an interpreter; \racketcode{namespace-require} implements the core of the module system; and \racketcode{read} and \racketcode{expand} collectively represent the complete macro expander. Thus, the expander linklet itself encapsulates the entire compilation phase required by Racket modules.

			\paragraph{}% 13
				Given the complexity and breadth of these functions—including extensive indirection, numerous cross-module references, and deeply nested loops—the resulting behavior severely complicates tracing and optimization by the meta-tracing JIT. Specifically, the meta-tracer and trace optimizer struggle to generate effective, reusable traces from such highly indirect and dynamic code paths. The \chapterRef{chapter:problem} is dedicated entirely to analyzing the performance implications of this complexity, detailing specific cases and investigating substantial runtime overhead caused by the introduction of these internally evaluated Racket sub-systems.

		\subsection{Interfacing with the Host Environment: IO \& Thread Linklets}

			\paragraph{}% 131
				Growing into a complete implementation also means the \gls{vm} must interact closely with the host environment. The thread and IO linklets handle this interaction. Specifically, the thread linklet provides user-level (green) threads, while the IO linklet implements comprehensive input/output capabilities ranging from file and port abstractions to high-level Racket primitives like \racketcode{write}. Collectively, these two linklets expose over 300 primitives, with the IO linklet explicitly relying on some primitives defined within the thread linklet.

			\paragraph{}% 14
				Some bootstrapping linklets—including thread and IO—require specialized runtime support because they assume particular functionalities or depend on external system-level operations. The thread linklet, for instance, requires the engines subsystem, and the IO linklet depends on the foreign function interface (FFI) to interact with the external \emph{rktio} C library. Additionally, these linklets define and expose numerous internal primitives previously implemented as stubs in Pycket (e.g., \racketcode{unsafe-start-atomic}). With the full integration of these linklets, such primitives now operate as originally intended.

			\subsubsection{Thread Linklet}

				\paragraph{}% 15
					The thread linklet provides the implementation for Racket's user-level (green) threads, supporting preemption through engines. At a high level, it defines useful Racket abstractions such as channels, threads, futures, and places. At a lower level, it implements essential synchronization and concurrency primitives including semaphores, custodians, and critical-section operations such as \racketcode{start-atomic} and \racketcode{end-atomic}. Because these features are entirely implemented within the linklet, any host runtime, such as Pycket, only needs to provide minimal primitive support required by the linklet to gain all these high-level functionalities for essentially free.

				\paragraph{}% 16
					Engines constitute one of the key primitive subsystems required by the thread linklet. Similar to an engine powering a vehicle, engines drive thread execution in Racket. Engines abstract the concept of timed preemption, providing a low-level mechanism for implementing time-sharing among arbitrary process abstractions. An engine is not itself a process abstraction; instead, it enables the time-sharing of any given computation. Within Racket threads, engines act as the computational core—each thread is associated with an engine that performs its assigned computation when provided with sufficient \emph{fuel}. Engines can be interrupted or blocked either by exhausting their allocated fuel or by external events, upon which a new engine is returned, encapsulating the remaining computation. This interrupted computation can subsequently be resumed by supplying additional fuel to the new engine instance \cite{enginesOriginal}.

				\paragraph{}% 17
					To support full preemption, Racket's engines differentiate between Racket-level continuations and host continuations through the concept of \emph{meta-continuations}. Meta-continuations represent segments of the host's continuation stack, delimited by prompt tags to define the dynamic extent of an engine's computation. When an engine is interrupted, the host continuation stack must be properly unwound to reflect the interruption of the corresponding meta-continuation. Conversely, resuming an engine requires reinstalling the saved meta-continuation onto the host continuation stack, restoring the execution state to the point where it was previously suspended.

				\paragraph{}% 18
					Pycket implements engines to provide the necessary primitive support for the thread linklet. However, Pycket does not support full preemption; rather, engines run to completion once started. This design choice stems from Pycket's underlying RPython framework, which assumes a single-threaded runtime protected by a Global Interpreter Lock (GIL). Consequently, full integration with OS-level preemption via pthreads is unnecessary and intentionally avoided. Instead, Pycket relies entirely on cooperative scheduling, where threads yield control explicitly through Racket-level primitives such as channels, semaphores, or synchronization points.

				\paragraph{}% 181
					With thread support enabled, all operations in Pycket run within the context of a Racket-level thread, initiated through the primitive \racketcode{call-in-main-thread}. Additional threads are spawned from this main thread, including a thread responsible for running a \gls{repl}, as well as any threads created during interactions within that \gls{repl}. To illustrate this clearly, consider the following example within Pycket's \gls{repl}:

				\begin{center}
					\begin{minipage}{0.5\textwidth}
						\begin{lstlisting}[style=bashstyle,frame=lines,numbers=none,basicstyle=\scriptsize\ttfamily\linespread{0.85}\selectfont]
				Welcome to Pycket v8.17.0.3
				> (define ch (make-channel))
				> (thread (lambda () (channel-put ch 3)))
				#<thread>
				> (channel-get ch)
				3
				>\end{lstlisting}
					\end{minipage}
				\end{center}

				\paragraph{}% 182
					While seemingly straightforward, the example above demonstrates important runtime behavior. The value transferred from the newly created thread to the main thread (running the \gls{repl}) via the channel illustrates explicit thread synchronization. Specifically, for a channel to facilitate such a transfer, both the sender and receiver threads must be suspended during the operation. In this example, the main thread initially suspends when it invokes \racketcode{channel-get}, awaiting the new thread to produce a value. The new thread runs and invokes \racketcode{channel-put}, placing a value into the channel and suspending itself. Subsequently, control returns to the main thread, retrieving the value from the channel. After the transfer, the newly created thread resumes briefly to complete its execution, and finally, the main thread resumes fully, returning control to the \gls{repl} prompt.

				\paragraph{}% 19
					The thread linklet also provides primitives utilized by other bootstrapping linklets, notably the IO linklet. As a result, the thread linklet is loaded earlier in Pycket's startup sequence, ensuring all threading primitives are available when the IO linklet is instantiated. This ordering guarantees that thread-related primitives required by IO operations—such as synchronization primitives and channels—are already established within Pycket's global primitive environment during the IO linklet's instantiation phase.

			\subsubsection{IO Linklet}

				\paragraph{}% 20
					The IO linklet implements Racket's extensive input/output capabilities, covering abstractions such as file systems, networking, and subprocess management. It exports essential high-level Racket primitives such as \racketcode{write}, as well as lower-level facilities for byte handling and data encoding. Internally, the IO linklet depends on the thread linklet, leveraging 49 primitives provided by it to implement thread-safe IO operations.

				\paragraph{}% 21
					To interact effectively with underlying operating system facilities, the IO linklet relies on a dedicated C library called \emph{rktio}. This standalone library provides a unified, portable interface to critical \gls{os}-level features—including file systems, networking, and subprocess management—exposing over 200 functions implemented in C. Rather than implementing OS abstractions directly within the linklet, the IO linklet delegates these operations to the rktio layer, maintaining a clean separation between Racket-level abstractions and low-level system interactions.

				\paragraph{}% 22
					To utilize the IO linklet, Pycket must load and expose rktio’s functionality to the Racket runtime. Since Pycket is implemented in RPython, it achieves this by creating an \gls{ffi} layer that bridges Racket code with the underlying rktio library. This integration is facilitated by automatically generating wrapper functions around every rktio C function from the provided rktio registry (rktio.rktl). These wrappers are exposed as primitives within Pycket’s global primitive environment, allowing the IO linklet to transparently invoke OS-level operations as if they were regular Racket primitives.

				\paragraph{}% 23
					Implementing this \gls{ffi} interface requires establishing clear type mappings between three distinct type domains: the native rktio C types, Pycket’s internal Racket-level types (\pycketcode{W_Object} and its derivatives), and the primitive RPython rffi types that map directly to C types during code generation. \tableRef{table:rktio-type-mappings} summarizes some notable examples. Pycket’s \gls{ffi} implementation must carefully manage these mappings to correctly handle memory operations, data conversion, and error propagation across language boundaries, ensuring accurate and efficient interaction between the Racket runtime and the underlying rktio library.

				\inputFigure{pycket}{rktio-type-mappings}

				\paragraph{}% 25
					In addition to simple type mappings, the interface between rktio and Racket involves a small but significant \gls{abi} convention for pointer representations. Specifically, rktio differentiates between memory regions that are opaque to Racket (referred to as \emph{ref}) and those transparent to Racket (referred to as \emph{*ref}). Opaque references represent pointers to memory whose layout and lifetime are fully managed by the C library, allowing Pycket to treat these as generic pointers (\pycketcode{rffi.VOIDP}). Transparent references, on the other hand, denote memory explicitly allocated, owned, and managed by Racket. For example, an rktio C function expecting a \emph{*ref char} input indicates it will mutate memory provided by Racket (such as a bytes object). Pycket must therefore ensure that any in-place mutations performed by the rktio function are accurately reflected in Racket-level representations, like \pycketcode{W_MutableBytes}, maintaining consistency between the C library's operations and Racket’s view of the data.

				\paragraph{}% 26
					With these type mappings and conventions in place, Pycket automatically generates the majority of the rktio \gls{ffi} wrapper functions directly from the rktio registry (rktio.rktl). Over 200 functions are automatically generated, while 26 additional primitives requiring special handling—such as complex struct access, specialized pointer dereferencing, or detailed error management—are implemented manually. To accommodate complex structures like \pycketcode{rktio_date_t}, Pycket includes specialized primitive definitions that form the connector layer. This connector layer explicitly manages struct layouts and pointer dereferencing, ensuring correct interactions between Pycket and rktio. Collectively, these automated and manually implemented primitives constitute a robust and comprehensive bridge, enabling seamless and efficient integration of the IO linklet into Pycket's runtime environment.


		\subsection{Language Utilities: Fasl \& Regexp Linklets}

			\paragraph{}% 27
				Racket also provides several utility linklets, notably the \emph{fasl} and \emph{regexp} linklets. The fasl linklet provides serialization and deserialization primitives (\racketcode{fasl->s-exp} and \racketcode{s-exp->fasl}), and the regexp linklet provides Racket's regular expression facilities. Compared to other bootstrapping linklets, these utility linklets are smaller but essential for Pycket’s frontend operation. For example, Pycket employs \racketcode{fasl->s-exp} to deserialize and load the initial bootstrapping linklets themselves.

			\paragraph{}% 28
				Previously, the functionalities provided by these utility linklets were manually implemented in RPython within Pycket. With the introduction of linklets, Pycket replaces its internal RPython implementations with Racket’s own implementations of fasl and regular expressions. This architectural shift allows for direct comparison between Pycket’s native RPython implementations and the corresponding Racket implementations in terms of meta-tracing performance, an analysis thoroughly explored in the performance evaluations presented in \chapterRef{chapter:problem}.

			\paragraph{}% 30
				These utility linklets further emphasize the modular, plug-and-play nature of the linklet-based architecture on Pycket. Once loaded, Racket’s native implementation becomes the active provider of the corresponding primitives, such as \racketcode{regexp-match}. Any subsequently instantiated linklets—including critical ones like the expander—transparently utilize these primitives without being aware of their underlying implementation. If, however, these utility linklets are not loaded, Pycket defaults gracefully to using its internal RPython implementations. Thus, the primitives exposed in Pycket's global primitive environment may originate either from RPython directly or from a loaded Racket linklet, without affecting the operation of dependent modules, given that both implementations are semantically equivalent.

	\section[\texorpdfstring{Growing Pycket into Full Racket}{Pycket as Full Racket}]{Growing Pycket into Full Racket}

		\inputFigure{pycket}{welcome-to-pycket}

		\paragraph{}% 1
			With all the bootstrapping linklets successfully loaded into its runtime, Pycket completes its transition into a fully operational, self-hosting Racket implementation. This marks the point where Pycket no longer relies on external Racket executables for critical runtime functionality, as it now internally provides the necessary components—such as macro expansion, module handling, and I/O—required to independently load and run Racket modules. Consequently, Pycket now fulfills the self-hosting hypothesis introduced earlier.

		\paragraph{}% 2
			Pycket's enhanced front-end supports a variety of execution modes, including directly running a provided \texttt{.rkt} source file or launching an interactive session to start the Racket \gls{repl}, as illustrated in \figref{fig:welcome-to-pycket}. These capabilities demonstrate that Pycket integrates core Racket functionality via bootstrapping linklets and exposes standard user-facing features. Furthermore, as new features are added to Racket itself, Pycket can readily incorporate them by simply running the corresponding Racket modules. This capability ensures Pycket remains robust and future-proof as a complete implementation of the language.

		\paragraph{}% 3
			The \gls{repl} provided by Pycket here is not custom-written; rather, it is the standard Racket \gls{repl} module (\texttt{racket/repl}), which Pycket dynamically loads by calling \pycketcode{dynamic-require} (provided by the expander linklet) at startup. Running the \gls{repl} necessitates loading the entire \texttt{racket/base} language immediately prior to executing Racket’s \gls{repl} implementation, thereby demonstrating Pycket's capability to independently expand and evaluate sophisticated Racket language modules directly within its runtime.

		\inputFigure{pycket}{figure-racket-base-loading-with-compiled}

		\paragraph{}% 4
			\figref{fig:racket-base-loading-with-compiled} illustrates loading the \texttt{racket/base} language during Pycket's startup with debug output enabled. Notably, the individual modules previously loaded explicitly via linklets no longer appear in this process (compare with \figref{fig:racket-base-loading-no-compiled}). This change occurs because Pycket leverages the \pycketcode{write} and \pycketcode{fasl} functionalities provided by Racket to serialize loaded modules into compiled \texttt{.zo} files. A \texttt{.zo} file is Racket-specific bytecode stored in fasl format. By supplying the Racket parameter \pycketcode{(use-compiled-file-paths compiled/pycket)} to the module system at boot time, Pycket loads these serialized modules directly into memory, bypassing the expensive linklet compilation and instantiation steps. This optimization significantly reduces the loading time of \texttt{racket/base}—from approximately 2 minutes down to about 7 seconds—achieving around a 95\% improvement in startup performance.

		\paragraph{}% 5
			Moreover, the availability of the expander linklet enables Pycket to execute higher-level languages built on top of \texttt{racket/base}, such as the full \texttt{\#lang racket}. For example, consider the Racket program shown in \figref{fig:racket-contract-example}, which utilizes Racket contracts. This program cannot run directly using only \texttt{racket/base}, as it requires additional modules from full Racket. However, Pycket successfully executes it, demonstrating its capability to load and evaluate sophisticated language-level constructs and further validating its status as a complete Racket implementation.

		\inputFigure{pycket}{racket-contract-example}

		\paragraph{}% 6
			With this, we conclude the first part of our thesis statement, the self-hosting hypothesis. We demonstrated concretely that it is possible to evolve a rudimentary interpreter into a full-featured, self-hosting implementation of a functional programming language on a meta-tracing \gls{jit} compiler. In the subsequent chapters, we will discuss how correctness of these developments is ensured, examine performance characteristics of the resulting system, investigate a fundamental performance issue uncovered in this self-hosting setup, and finally propose and evaluate several approaches to address this issue.


% \begin{figure}[h!]
%   \centering
% \includegraphics[scale=0.3]{img/new-pycket-yatay}
% \caption{Pycket now uses the functionalities from the expander linklet to expand and run a given module.}
% \label{fig:new-pycket}
% \end{figure}


