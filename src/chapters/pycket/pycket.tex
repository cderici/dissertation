\chapter[\texorpdfstring{LANGUAGE-ENHANCED RUN-TIMES}
                          {4. Pycket as Full Racket}]{LANGUAGE-ENHANCED RUN-TIMES}
	\label{chapter:pycket}

	\begin{chaptersynopsis}

        Languages can influence the capabilities of the run-time they are implemented on.

		Concrete realization of self-hosting of Racket on Pycket.

        \vspace{2em}

        Sections:
		\begin{itemize}
			\item Implementing Linklets on Pycket

			This is how Pycket is enhanced to interface with linklets. Representation choices, etc.

			\item Enhancing Run-time with Bootstrapping Linklets

			Importing functionality straight from the language in the form of linklets allows the language to shape the behavior of the run-time.

			\item From Rudimentary Interpreter to Full Language Implementation

			Pycket becomes a full working Racket, proving the self-hosting hypothesis.

		\end{itemize}
    \end{chaptersynopsis}

	\begin{paragraph-here}
		Pycket is a suitable host and environment for proving the self-hosting hypothesis.
	\end{paragraph-here}

	\section[\texorpdfstring{Implementing Linklets on Pycket}{Linklets on Pycket}]{Implementing Linklets on Pycket}

		\begin{mainpoint}
			This is how Pycket is enhanced to interface with linklets. Representation choices, etc.
		\end{mainpoint}

		Runtimes can represent linklets however they want.

		Chez Scheme represents them as functions, here's how Pycket represents and implements them.

	\section[\texorpdfstring{Enhancing Run-time with Bootstrapping Linklets}{Bootstrapping Linklets}]{Enhancing Run-time with Bootstrapping Linklets}
		\begin{mainpoint}
			Importing functionality straight from the language in the form of linklets allows the language to shape the behavior of the run-time.
		\end{mainpoint}

		\subsection{Interfacing with the Compiler: Expander Linklet}

		\subsection{Interfacing with the Host Environment: IO \& Thread Linklets}
			\begin{todo}[TODO]
				rktio
			\end{todo}
			\begin{todo}[TODO]
				engines
			\end{todo}
		\subsection{Language Utilities: Fasl \& Regexp Linklets}

	\section[\texorpdfstring{From Rudimentary Interpreter to Full Language Implementation}{Pycket Elevated}]{From Rudimentary Interpreter to Full Language Implementation}

		\begin{mainpoint}
            Pycket becomes a full working Racket, proving the self-hosting hypothesis.

			\textit{Self-hosting of full-scale functional languages on meta-tracing just-in-time (JIT) compilers is achievable.}
        \end{mainpoint}

        \begin{todo}
			A rudimentary interpreter can be bootstrapped into a full language implementation by importing linklets into the run-time.
		\end{todo}

		\begin{todo}[Import]
			Section 3.3 Making Pycket an Independent Self-Hosting Racket of the proposal document is relevant for this section.
		\end{todo}



		Exploiting the increased portability in Racket, here we turn the
Pycket from being a rudimentary compiler to an implementation of
Racket, independent of the Racket binary. First step is building the
\emph{linklets} layer by implementing the \emph{compile-linklet} and
\emph{instantiate-linklet} functions in RPython, thereby allowing
Pycket to process and evaluate linklets. The \emph{compile-linklet}
takes an s-expression for a linklet to produce a linklet object as
described in \secref{subsec:linklets-semantics}. It makes several
passes on the s-expression to determine the identifiers that are
defined within the linklet and the ones that are mutated. Then it
processes the imported and exported identifiers to produce the Import
and Export run-time objects, freshly generating identifiers (gensym)
as needed. Then it makes a final recursive pass to produce the RPython
AST for the body, and puts all the information within a linklet
object. The \emph{instantiate-linklet} on the other hand, carries out
the steps for running a linklet object. First it processes the Import
and Export objects and collects and creates variables as necessary,
and prepares the target instance (i.e. creates a new instance if it's
a regular instantiation) and interprets the body, effectively
implementing the $\longrightarrow_{\beta p}$ reduction in the \figref{fig:reduction}
which will be transitively applied by the interpreter loop.

After implementing the \emph{compile-linklet} and
\emph{instantiate-linklet}, the next step is to load the bootstrapping
linklets into the run-time as described in
\secref{subsec:racket-expand}. For simplicity, we are only concerned
with loading the \emph{expander} linklet, as it's the most essential
one for bootstrapping (also it's the biggest one). Loading the other
linklets are essentially the same process (modulo the run-time support
they need).

% \begin{figure}[h!]
%   \centering
% \includegraphics[scale=0.3]{img/new-pycket-yatay}
% \caption{Pycket now uses the functionalities from the expander linklet to expand and run a given module.}
% \label{fig:new-pycket}
% \end{figure}


As we described in \secref{subsec:racket-expand}, the expander linklet
is generated offline by running the expander on itself using the
Racket's executable, and serialized as an s-expression. Pycket, then
reads this s-expression and runs \racketcode{compile-linklet} to produce a
linklet object, which is then instantiated using the
\racketcode{instantiate-linklet}, and the functions from the expander
linklet is load into the run-time. At this point, Pycket has all the
functionalities implemented and exported by the expander, such as
read, expand, eval etc., in its run-time to call.

Having the expander and all the functions that it provides in its
run-time allows Pycket to run a wide variety of Racket programs and
easily implement run-time functionalities. For example, a very simple
implementation of our running example top-level repl can directly use
Racket's read, expand and eval functions, as shown in
\figref{fig:repl-rpython}. The \verb|call_racket| is a wrapper that
handles the setup to call Racket functions within the CEK loop.

Moreover, since the expander provides the implementations of Racket's
macro-system and the module-system, Pycket is able to run programs
that are in languages built on top of Racket core, such as
\emph{racket/base}. Therefore, instead of implementing the top-level
repl itself, Pycket can directly run Racket's own repl, which is
implemented as a \emph{racket/base} program, through the
\racketcode{dynamic-require} function exported by the expander.


These programs, including the bootstrapping linklets themselves often
require additional run-time support such as \emph{correlated} objects
(which are syntax objects without the lexical-content information),
Racket-level exception handling and primitive functions. For example,
running Racket's repl requires delimited continuations. Also handling
exceptions in Racket code in the run-time requires Racket-level
exception handling on Pycket. The existing Pycket implementation has a
rudimentary exception handling mechanism via transforming the
Racket-level exceptions (which are implemented by Racket structs) into
the RPython exceptions. However, having Racket-level exceptions
requires the run-time to support installing Racket-level handlers via
continuation-marks and dynamically pass the Racket-level exceptions to
the appropriate handlers.

Moreover, the Racket itself (including the bootstrapping linklets)
generally relies on a large number ($\sim$1500) of primitives, aroud 900
of which we already have in the existing Pycket implementation. An
additional $\sim$200 primitives need to be implemented, along with the
addition of the run-time liblaray \texttt{\#\%linklet} that will
contain 32 linklet related functions including the
\emph{compile-linklet}, \emph{instantiate-linklet} and
\emph{instance-variable-value}.