\chapter[\texorpdfstring{PYCKET AS A FULL RACKET RUN-TIME: VALIDATION \& CORRECTNESS}
                          {5. Validation \& Correctness}]{PYCKET AS A FULL RACKET RUN-TIME: VALIDATION \& CORRECTNESS}

	\label{chapter:validation}

	\begin{chaptersynopsis}
		Bootstrapped Racket on Pycket works, and is correct and complete.
	\end{chaptersynopsis}


	\section{Correctness by Construction}

		\begin{mainpoint}
			Pycket is correct wrt Racket.
		\end{mainpoint}

		\subsection{Ensuring Correctness of the Linklet Layer}
			\begin{mainpoint}
				We have 100\% coverage on the operational semantics as unit tests.

				Additionally, we symmetrically used the same set of tests with the redex model that realize the operational semantics.
			\end{mainpoint}

			\begin{todo}[Import]
				Use section 1.4 "Testing the model" from the Quals written document for testing linklets.
			\end{todo}

		\begin{paragraph-here}
			We add untouched Racket code on top of original Pycket to expand functionality.

				Correctness is preserved.
		\end{paragraph-here}

		\begin{paragraph-here}
			Overall Correctness of the System is ensured by the Old and New Unit and end-to-end/integration tests.
		\end{paragraph-here}

	\section{Completeness under Self-Hosting}

		\begin{mainpoint}
			Pycket can evaluate any \#lang that Racket evaluates.
		\end{mainpoint}

		\subsection{From \#\%kernel to \#lang racket/base}
			\begin{mainpoint}
				Racket is a programming language programming language.

				As long as you have the expander to reduce the given language to \#\%kernel, you're golden.
			\end{mainpoint}

		\subsection{Beyond \#lang racket/base}
			\begin{mainpoint}
				Being able to implement \#lang racket/base is a good springboard to implement larger languages.
			\end{mainpoint}