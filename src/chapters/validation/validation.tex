\chapter[\texorpdfstring{PYCKET AS A FULL RACKET RUN-TIME: CORRECTNESS \& COMPLETENESS}
                          {5. Validation}]{PYCKET AS A FULL RACKET RUN-TIME: CORRECTNESS \& COMPLETENESS}

	\label{chapter:validation}

	\begin{chaptersynopsis}[Chapter Synopsis - \emph{Chapter Content: 10\%}]

        We demonstrate three things in this chapter:

        1. Self-hosted Racket on Pycket works.

        2. Semantics of the self-hosted Racket on Pycket are identical to the semantics of Racket on Chez Scheme (Racket's own default runtime).

        3. You can evaluate on Pycket anything you can express in Racket.

        \vspace{2em}

        Sections:
		\begin{itemize}
			\item Correctness by Construction

                Semantics of the self-hosted Racket on Pycket are identical to the semantics of Racket on Chez Scheme (Racket's own default runtime).
			\item Completeness under Self-Hosting

                Pycket can evaluate any \#lang that Racket can evaluate.
		\end{itemize}
    \end{chaptersynopsis}

	\inputFigure{validation}{big-example}

	\section[\texorpdfstring{Correctness by Construction}{Correctness}]{Correctness by Construction}

		\begin{mainpoint}
			Pycket is correct wrt Racket.
		\end{mainpoint}

		\subsection{Randomized Testing for Linklet Semantics}
			\begin{mainpoint}
				We have 100\% coverage on the operational semantics as unit tests.

				Additionally, we symmetrically used the same set of tests with the redex model that realize the operational semantics.
			\end{mainpoint}

			\begin{todo}[Import]
				Use section 1.4 "Testing the model" from the Quals written document for testing linklets.
			\end{todo}

		\begin{paragraph-here}
			We add untouched Racket code on top of original Pycket to expand functionality.

				Correctness is preserved.
		\end{paragraph-here}

		\begin{paragraph-here}
			Overall Correctness of the System is ensured by the Old and New Unit and end-to-end/integration tests.
		\end{paragraph-here}

	\section[\texorpdfstring{Completeness under Self-Hosting}{Completeness}]{Completeness under Self-Hosting}

		\begin{mainpoint}
			Pycket can evaluate any \#lang that Racket evaluates.
		\end{mainpoint}

		\subsection{From \#\%kernel to \#lang racket/base}
			\begin{mainpoint}
				Racket is a programming language programming language.

				As long as you have the expander to reduce the given language to \#\%kernel, you're golden.
			\end{mainpoint}

		\subsection{Beyond \#lang racket/base}
			\begin{mainpoint}
				Being able to implement \#lang racket/base is a good springboard to implement larger languages.
			\end{mainpoint}