\chapter[\texorpdfstring{PYCKET AS A FULL RACKET RUN-TIME: CORRECTNESS \& COMPLETENESS}
                          {5. Validation}]{PYCKET AS A FULL RACKET RUN-TIME: CORRECTNESS \& COMPLETENESS}

	\label{chapter:validation}

	\begin{chaptersynopsis}[Chapter Synopsis]
		\footnotesize

        We demonstrate three things in this chapter:

        1. Self-hosted implementation of Racket on Pycket works.

        2. Semantics of the self-hosted Racket on Pycket are identical to the semantics of Racket on Chez Scheme (Racket's own default runtime).

        3. You can evaluate on Pycket anything you can express in Racket.

        \vspace{2em}

        Sections:
		\begin{itemize}
			\item Correctness by Construction

                Semantics of the self-hosted Racket on Pycket are identical to the semantics of Racket on Chez Scheme (Racket's own default runtime).

				\textasciitilde 550 Test Suites.

				Same tests against Racket, Pycket \& Redex Models.
				Randomized testing for the linklet Redex model.
			\item Completeness under Self-Hosting

                Pycket can evaluate any \#lang that Racket can evaluate.
		\end{itemize}
    \end{chaptersynopsis}

	\paragraph{}% 1
		In this chapter, we demonstrate that Pycket's self-hosted implementation of Racket is both correct and complete with respect to the original Racket implementation. Correctness here means that the semantics provided by Pycket exactly match those provided by Racket's default runtime on Chez Scheme, and completeness means that any program expressible in Racket can be successfully evaluated by Pycket. Establishing these two properties is essential to demonstrate that Pycket can indeed serve as a fully functional drop-in replacement for Racket.

	\paragraph{}% 2
		As the first step of our validation, we use a moderately sized program, shown in \figref{fig:pycket-works-big-example}, as an example to illustrate that Pycket's runtime behavior is indistinguishable from Racket running on Chez Scheme. This program works correctly on both implementations.

	\inputFigure{validation}{big-example}

	\paragraph{}% 3
		For the program in \figref{fig:pycket-works-big-example} to correctly run on Pycket, multiple essential components must function together seamlessly. At the outset, Pycket must support the full loading and evaluation of a \racketcode{#lang racket} module, which alone involves significant runtime support for nearly all Racket core values and data structures. This includes handling numerical values, strings, symbols, hash tables, vectors, boxes, and particularly structs, which are fundamental to syntax objects used extensively by Racket’s macro expander. Indeed, before considering linklets, Pycket must already have robust runtime support for syntax objects--heavily reliant on Racket structs--to correctly handle macros such as \racketcode{define-syntax}, \racketcode{syntax-case}, and phase-level forms like \racketcode{for-syntax}. Moreover, because the macro expander itself utilizes linklets to manage dependencies across different compilation phases, Pycket must correctly instantiate and evaluate these linklets, ensuring accurate handling of variable binding, closures, and across-phase interactions. Additionally, to correctly evaluate advanced control-flow mechanisms such as \racketcode{shift0} and \racketcode{reset0}, Pycket's runtime must provide reliable and efficient support for delimited continuations. Further, the runtime must comprehensively support primitive operations required by the fully expanded program, including boxing/unboxing operations, structural equality checks, and errors. Since Pycket successfully executes this substantial program, it demonstrates the correctness and completeness of Pycket’s implementation of all these interdependent components.

	\paragraph{}% 4
		The semantic equivalence between Racket running on Chez Scheme and Pycket ultimately relies on Pycket's correct handling of the linklet semantics, as well as the correctness of its primitive operations and core data structures exposed to linklets. Given these foundational components, completeness naturally follows from Racket’s layered language architecture, in which richer language constructs and functionalities are incrementally constructed atop more primitive layers. The following two sections detail the evidence supporting both correctness and completeness.

	\section[\texorpdfstring{Correctness by Construction}{Correctness}]{Correctness by Construction}

		\paragraph{}% 1
			Correctness by construction in Pycket begins with an already comprehensive and correct implementation of Racket's core values and data structures. This includes the full numeric tower, symbols, hash tables, vectors, and structs, as described in detail in \cite{pycketmain}. We build our high-level correctness argument upon this robust correctness of these fundamental elements.

		\inputFigure{validation}{test-suites-table}

		\paragraph{}% 2
			Pycket ensures correctness of its primitive operations through an extensive suite of unit tests. \tableRef{table:validation-test-suites-table} shows the number and scope of Pycket's test suites, forming a total of 549 suites, each containing numerous individual test cases. Among all the baseline semantics, these suites comprehensively cover significant runtime components such as structs, hashes, equality, hidden classes, impersonators, regular expressions, \gls{ast} handling, and linklets.

		\paragraph{}% 3
			We considered leveraging Racket’s own comprehensive test suites to further demonstrate Pycket's correctness. However, since Racket’s testing infrastructure heavily relies on external tooling such as \racketcode{raco} (e.g., \racketcode{raco test -p racket-test-core}), integrating it directly into Pycket's testing workflow would exceed the scope of our study. Instead, we focus on targeted, manually adapted tests that precisely verify the critical aspects of our implementation.

		\paragraph{}% 4
			A crucial component of ensuring Pycket’s correctness is validating its implementation of the linklet semantics. The correctness of linklets is particularly important because linklets constitute the primary compilation units upon which Racket's macro expander and module system are constructed. In the rest of the section, we explain the methodology employed to ensure that Pycket correctly implements these semantics, aligning precisely with the semantics defined by Racket itself.

		\paragraph{}% 5
			Recall that the operational semantics we presented for linklets in \chapterRef{chapter:linklet} were developed using PLT Redex, which allows us to create an executable model of these semantics. The complete linklet model is provided in \appendixRef{appendix:linklet-semantics-model-redex-code}. In the remainder of this section, we describe how we leveraged this executable model to validate Pycket’s implementation of linklet semantics.

		\paragraph{}% 6
			To ensure correctness of Pycket's implementation of linklet semantics, we manually adapted Pycket’s own linklet tests into the \racketcode{program} form defined in our PLT Redex model (refer to \figref{fig:linklet-program-source-language} for the definition of the program form). This adaptation enabled us to create a symmetric testing setup, allowing us to run each test across Racket itself, Pycket, and the Redex model simultaneously. Running these tests concurrently ensures that all implementations consistently agree on the expected operational semantics.

		\inputFigure{validation}{test-equal-listing}

		\paragraph{}% 7
			An example test case (adapted from Pycket's suite into the Redex model) for the correctness of linklet semantics is shown in \figref{fig:test-equal-listing}. Here, the linklet \textbf{l2} exports a closure \racketcode{g}, which references a variable \racketcode{y} initially defined as 10 and subsequently mutated to 50 within the linklet (line 9). Since \racketcode{y} is exported and mutated within \textbf{l2}, the compiler treats it as a linklet variable. Similarly, in linklet \textbf{l3}, the identifier \racketcode{y} is exported without a corresponding local definition and is also mutated (line 11), thus compiled as a linklet variable as well. After compilation, we first instantiate \textbf{l2} with the target instance \racketcode{t1} (line 15), populating it with exported variables including the closure \racketcode{g} and the variable \racketcode{y}. We then instantiate \textbf{l3} using the same target instance \racketcode{t1}. Any reference to \racketcode{y} within \textbf{l3} thus accesses and modifies the shared linklet variable in the target instance. At the time \textbf{l3} is instantiated, the value of \racketcode{y} is 50, but before invoking the closure \racketcode{g}, \textbf{l3} updates \racketcode{y} to 200 (line 11). When the closure \racketcode{g} is finally invoked (line 12), it yields a result of 201, highlighting the use of the updated linklet variable \racketcode{y}, despite lexical scoping; otherwise, without linklet variables, the closure would have used the original lexical environment in which \racketcode{y} was 10, returning 11. Notably, after this invocation, the variable \racketcode{y} in the target instance \racketcode{t1} remains modified to 200, which can be confirmed by invoking \racketcode{(instance-variable-value t1 'y)}, demonstrating correct handling of linklet variables across instances.

		\subsection{Randomized Testing for Linklet Semantics}

			\paragraph{}% 8
				In addition to manually adapted test cases, we also utilized PLT Redex's support for randomized testing to further verify the correctness of our implementation of linklet semantics. Redex enables generation of random terms according to a provided grammar, which are then checked against a correctness predicate. In our setup, we leverage randomized testing separately for two executable models: the core language model (\gls{lkl}) and the linklets model. For each randomly generated term, we asked whether the semantics defined by our model agreed with the actual implementation provided by Racket.

			\paragraph{}% 9
				Specifically, we tested two correctness predicates, implemented as meta-functions within our Redex models: \racketcode{eval-rc=racket-core}, defined for the \gls{lkl} language (\figref{appendix:linklet-kernel-language-figure}), and \racketcode{eval-prog=racket-linklets}, defined for the linklets language (\figref{fig:linklet-grammar}). The meta-function \racketcode{eval-rc=racket-core} receives a randomly generated term in the \gls{lkl} language, then simultaneously evaluates it using both the Redex model's evaluator and the actual Racket interpreter (on which PLT Redex itself runs). The results are compared to verify that both implementations produce identical outcomes starting from an empty environment. Similarly, the \racketcode{eval-prog=racket-linklets} predicate performs an analogous check for the linklets language, ensuring semantic consistency between our linklets model and Racket’s own implementation.

			\paragraph{}% 10
				During randomized testing, we encountered a notable challenge due to the difference between the Redex language grammars and Racket's actual syntax. Specifically, the grammars for both the \gls{lkl} and the linklets languages allow the explicit representation of certain runtime entities--most notably closures--that do not have direct syntactic counterparts in Racket itself. Closures in both Racket and the \gls{lkl} language consist of formal parameters, a function body, and the environment captured at closure creation time. However, Racket’s surface syntax does not allow for explicit construction of closures, as environments lack a direct syntactic representation. Consequently, randomly generated terms containing explicit closures (e.g., \racketcode{(closure x (+ x y) ((y cell123)))}) would lead to runtime exceptions when evaluated by Racket’s interpreter, even though they were successfully evaluated by our Redex model.

			\inputFigure{validation}{restricted-lkl-expression-grammar}

			\paragraph{}% 11
				To mitigate this issue and reduce the frequency of errors caused by such randomly generated cases, we slightly restricted the grammar used by the random term generator. Specifically, we removed explicit closure terms (the non-terminal \racketcode{c}) from the \gls{lkl} grammar for randomized testing, ensuring that all randomly generated terms have direct syntactic representations in Racket. The restricted grammar for these \gls{lkl} expressions is shown in \figref{fig:restricted-lkl-expression}. Using this restricted grammar, we performed randomized testing with 1000 freshly generated terms, and successfully confirmed that our \gls{lkl} model agrees with the actual Racket implementation. As we discuss later, the grammar restriction for the linklets language required a slightly more involved adjustment.

			\paragraph{}% 13
				Testing the linklets model required a more elaborate approach compared to testing the \gls{lkl} model, for two main reasons. First, because the terms are randomly generated, the linklets model needed to robustly handle numerous runtime corner cases inherent to module-like constructs. Imagine randomly generating complete Racket modules using the full Racket grammar and evaluating them: while such modules would be syntactically valid, they would frequently yield runtime errors due to undefined identifier references, function arity mismatches, or type errors--such as attempting arithmetic between incompatible types. Second, while terms generated from the \gls{lkl} grammar are directly evaluable in both Racket and the Redex model without transformation, terms generated from the linklets grammar include forms (such as the top-level \racketcode{program} form) that are not directly recognized by Racket’s evaluator, thus necessitating additional steps to enable evaluation in Racket.

			\inputFigure{validation}{initial-checks-for-test-redex}

			\paragraph{}% 14
				To handle the first challenge, where many randomly generated inputs were semantically invalid, we had two potential solutions. The first option was to employ a mechanism provided by Redex, such as the \racketcode{#:prepare} argument in \racketcode{redex-check}, which allows filtering generated terms to discard those that are semantically invalid before applying the predicate. The second option was to directly embed checks within the linklets evaluator model to explicitly identify and handle these error cases. While the first approach would have provided a streamlined testing, we chose the second approach, as it aligns our model more closely with Racket's own implementation. Specifically, we inserted preemptive checks into our linklets evaluator to detect and appropriately handle common semantic errors. The downside of this approach is that a significant portion of randomly generated test cases end up being rejected by both our model and Racket's implementation, thus reducing the proportion of interesting cases that fully exercise the instantiation semantics. However, this trade-off ensures that our model accurately mirrors the behavior of Racket's own system, which detects such semantic errors during the \racketcode{compile-linklet} phase, where linklet objects are constructed from their s-expression representations. In contrast, our Redex model does not have a separate compilation phase; instead, the representation of the object is the object itself, as evaluation directly operates on Redex meta-terms. The details of these checks are shown in \figref{fig:restricted-lkl-eval-prog}.

			\inputFigure{validation}{convert-program-to-racket}

			\paragraph{}% 15
				To address the second issue--where randomly generated terms from the linklets grammar include forms not directly evaluable by Racket--we implemented a straightforward converter. This converter rewrites program forms from our linklets language into equivalent Racket syntax, utilizing standard constructs such as \racketcode{let} \& \racketcode{define}, as demonstrated in \figref{fig:convert-program-to-racket}. This transformation allows us to evaluate generated terms directly using Racket's evaluator, thereby enabling accurate semantic comparisons between Racket and our Redex model.

			\paragraph{}% 16
				As noted previously for testing the \gls{lkl} model, another complication in random testing arises from generating terms that include entities without explicit representations in Racket. Just as we addressed this issue by restricting the \gls{lkl} grammar for randomized tests, we similarly restricted the grammar for the linklets language. Specifically, we removed explicit closure values and linklet instance terms from the grammar. Linklet instances, like closures, contain runtime-specific entities (variables mapped to cells), which cannot be directly expressed in Racket's syntax. In place of these instance literals, our tests instead use linklet literals combined with the \racketcode{instantiate} form, producing the desired linklet instances at runtime. The resulting restricted grammar for randomly generated linklet programs is shown in \figref{fig:restricted-linklet-program}.

			\inputFigure{validation}{restricted-linklet-program-grammar}

			\paragraph{}% 17
				Using this adjusted grammar, and increasing our coverage by running tests with 2000 freshly generated terms each time, we successfully confirmed that our linklets model aligns precisely with Racket's own linklet implementation (\racketcode{racket/linklet}). This consistency between the Redex model and Racket’s implementation confirms the correctness of Pycket’s implementation of linklet semantics.

			\paragraph{}% 18
				In summary, by combining the correctness of Pycket's foundational core values, data structures, and primitive operations with validation of its linklet semantics via targeted and randomized testing, we have ensured that Pycket accurately evaluates the linklet forms generated by Racket’s macro expander. The targeted tests--adapted to a symmetric setup--were simultaneously executed on Racket, Pycket, and our Redex model, further strengthening our confidence in Pycket's correctness by ensuring consistency across all implementations. Moreover, incorporating untouched Racket code into Pycket, code already validated on Racket’s own runtime, effectively serves as integration testing. This provides additional assurance of correctness for any runtime that imports these functionalities via linklets.

	\section[\texorpdfstring{Completeness under Self-Hosting}{Completeness}]{Completeness under Self-Hosting}

		\paragraph{}% 1
			Racket is not only a programming language, but also a language‐building platform. It provides an integrated framework--including a module system, a hygienic macro expander with syntax objects, customizable parsers via \racketcode{#lang}, and a rich runtime--that makes it straightforward to create new domain‑specific or general‑purpose languages. As Felleisen et al. put it, Racket’s design “supports a smooth path from relatively simple language extensions to completely new languages” \cite{racket:create-langs}.

		\paragraph{}% 2
			Because Pycket correctly evaluates Racket's expander--imported among the bootstrapping linklets--it inherits this full language‑construction capability. The expander takes any \racketcode{#lang} module and translates it down to linklets, which Pycket then evaluates using the primitive and linklet semantics we've already validated. Thus, any language built on Racket--for instance, Typed Racket, Scribble, or even educational languages--becomes directly executable in Pycket (a meta-tracing \gls{jit} compiler), just as it is in native Racket.

		\paragraph{}% 3
			In essence, Pycket’s ability to successfully import and evaluate the expander forms the minimal sufficient condition for full Racket compatibility. From \racketcode{#\%kernel} through \racketcode{#lang racket/base} up to arbitrary \racketcode{#lang} dialects, all macro phases, parser extensions, and runtime behaviors--including delimited continuations, contracts, closures, and error reporting--are retained. This validates Pycket as a complete drop‑in Racket runtime.

		\paragraph{}% 4
			In conclusion, this chapter has demonstrated that Pycket is both correct and complete as a fully self-hosting Racket run-time. We established correctness through foundational unit testing, careful validation of linklet semantics using targeted and randomized tests, and consistency checks across Pycket, Racket, and our executable Redex model. Completeness follows naturally from Racket’s layered architecture, which Pycket accurately supports via correct evaluation of the imported macro expander and the module system. Ultimately, the substantial example provided at the beginning of this chapter (\figref{fig:pycket-works-big-example}) decisively confirms our initial claim: Pycket successfully executes non-trivial \racketcode{#lang racket} programs, thereby validating its role as a drop-in, functionally transparent Racket run-time.
