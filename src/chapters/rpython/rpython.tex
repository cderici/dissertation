% Tracing JITs and Meta-Tracing Frameworks
% Tracing Just-in-time (JIT) Compilers and Meta-tracing
\chapter[\texorpdfstring{META-TRACING JUST-IN-TIME (JIT) COMPILERS}
                          {RPython \& Meta-tracing}]{META-TRACING JUST-IN-TIME (JIT) COMPILERS}
    \label{chapter:rpython}

    \begin{chaptersynopsis}

        We do three things in this chapter:

        1. Thesis statement refers to meta-tracing JITs, so we explain what a meta-tracing JIT compiler is.

        2. Introduce RPython as a language, and a framework that Pycket is built on.

        3. Provide context for meta-tracing-related technical discussions in the rest of the study. traces, warmup, optimizations, runtime feedback, etc.

        \vspace{2em}

        Sections:
		\begin{itemize}
			\item Meta-tracing \& RPython Framework

                What problem does PyPy solve? How do they capture loops in user programs? JIT drivers, interpreter hints, green/red variables etc.
			\item Pycket: A Rudimentary Racket Interpreter Built on RPython Framework

                Primer on Pycket's fundamentals. It's language RPython, CEK core, and how it is built.
			\item Trace Optimizations \& Runtime Feedback

                Traces are the real performance currency. Relevant optimizations and runtime feedback (e.g promote, escape analysis, warmup, etc.).
		\end{itemize}
    \end{chaptersynopsis}

    \paragraph{}%
        In this chapter, we have three main objectives. First, we explain the concept of a meta-tracing Just-in-Time (JIT) compiler, clearly connecting it to our thesis statement presented earlier, that efficient self-hosting of full-scale functional languages is achievable using meta-tracing JIT compilation. Second, we introduce the RPython framework, detailing its role as the foundational technology upon which Pycket is constructed, including aspects of translation and toolchain workflow. Lastly, we outline key concepts related to RPython and meta-tracing—such as trace formation, optimization strategies, and runtime feedback—which serve as essential context for technical discussions and evaluations relating to performance in \chapterRef{chapter:problem} and \chapterRef{chapter:solution}.

    \paragraph{}%
        A Just-in-Time (JIT) compiler dynamically compiles frequently executed parts of a program at runtime, interleaving compilation with interpretation. Unlike Ahead-of-Time (AOT) compilers, which compile the entire program beforehand, JIT compilers selectively target hot paths identified through low-overhead profiling. When the interpreter repeatedly executes a particular sequence of instructions, the JIT compiler pauses interpretation momentarily, compiles and optimizes this sequence, and subsequently executes the optimized code whenever the same instruction path recurs \cite{dynamo}.

    \paragraph{}%
        JIT compilers have proven particularly effective for dynamic language virtual machines (VMs), generally adopting either method-based or trace-based approaches. Method-based compilers optimize frequently invoked methods, whereas trace-based compilers focus on frequently executed loops \cite{survey:05,jit-history:03}. This dissertation specifically focuses on trace-based compilation. Tracing JITs generate optimized machine code by tracing and compiling execution paths under two fundamental assumptions\cite{pypy-main}:

    \begin{enumerate}
        \item Programs spend most of their execution time in loops.
        \item Iterations of the same loop often follow similar execution paths.
    \end{enumerate}

    \inputSub{rpython}{fig-trace}

    \paragraph{}%
        To exploit these assumptions, tracing JIT compilers identify certain execution paths as \emph{hot loops}. Upon detecting a hot loop, the interpreter pauses evaluation to compile this loop into a \emph{trace}, subsequently using the optimized trace whenever the same path is executed again. A trace is a linear sequence of instructions with a single entry and potentially multiple exit points. \figref{fig:trace} illustrates a simplified trace, with inputs $p0$ and $p1$, comprising a preamble (due to loop unrolling) and an inner loop that jumps back to itself. \emph{Guards} within a trace define its exit points, performing runtime checks to ensure that conditions remain consistent with the initial tracing context and handling conditions for loop termination or deviation.

    % \begin{paragraph-here}
    %     The tracer is a component of a JIT compiler that is responsible for tracing the execution of the program and generating traces.
    % \end{paragraph-here}

    % \begin{paragraph-here}
    %     Whenever a hot loop is detected, the tracer generates a trace that captures the execution of the loop.

    %     For a given program, a time that it takes for the JIT to find and trace all the hot loops is called the warmup time.
    % \end{paragraph-here}

    \section[\texorpdfstring{Tracing vs Meta-tracing in JIT Compilation}{Tracing vs Meta-tracing}]{Tracing vs Meta-tracing in JIT Compilation}
        \begin{mainpoint}
            The problem that PyPy solves:

            Rather than the loops in the interpreter being evaluated, meta-tracing manages to capture the hot loops in the user program being interpreted.
        \end{mainpoint}

        \paragraph{}%
            When a tracing JIT compiler is applied to a language interpreter, it typically ends up tracing the loops of the interpreter itself—rather than the loops in the user program that the interpreter is executing. To understand why this is a problem, we need to distinguish three roles: the language interpreter (the implementation of the language semantics), the tracing interpreter (or meta-tracer), and the user program (the input being evaluated by the language interpreter). The meta-tracer effectively executes the language interpreter step-by-step, carrying out its operations and observing their effects. Without additional guidance, it has no visibility into the boundaries or semantics of the user-level code being interpreted, and so it naturally captures hot paths in the language interpreter logic rather than in the user program.

        \paragraph{}%
            Within a language interpreter, the primary dispatch loop is typically the most significant loop. For a functional programming language, this usually manifests as a recursive descent loop, evaluating sub-expressions iteratively. However, while evaluating a user program, this loop structure in the language interpreter fundamentally violates one of the core assumptions of tracing JIT compilers—that several iterations of a hot loop tend to follow similar code paths—because it is uncommon for an interpreter to repetitively evaluate the exact same operation or expression in succession. Rather, loops in the user program often unfold as different branches taken through the language interpreter’s dispatch logic.

        \paragraph{}%
            Furthermore, by nature, tracing captures the execution paths where a program spends most of its running time. In the scenario of applying a tracing JIT to a language interpreter, however, the interpreter's dispatch loop itself is typically not the most essential computational path. Instead, the loops within the user program represent the true computational hot spots. Thus, what we seek from a tracing JIT in this context is to identify and optimize the loops of the user program, not the loops within the interpreter. This is precisely the problem addressed by \emph{meta-tracing}, as defined in PyPy: it successfully redirects the tracing JIT compiler to identify and optimize the hot loops of the user program, rather than the interpreter logic itself \cite{pypy-main}.

        \paragraph{}%
            A loop in the context of execution is essentially a backward jump to an instruction or a logical position (e.g., a previously seen program counter). The meta-tracer, lacking any intrinsic semantic knowledge of the language interpreter it is running, cannot inherently detect such loops occurring within the interpreted user program. To overcome this, meta-tracing frameworks expose an API through which the language interpreter explicitly communicates the occurrence and location of loops within the user program being interpreted. Specifically, the interpreter defines a logical \emph{program counter} composed of interpreter-specific variables. For instance, a bytecode interpreter might represent this counter by combining the current bytecode index with other execution-specific information. When the meta-tracer repeatedly observes the same logical program counter value, it infers the presence of a loop within the user program.

        \paragraph{}%
            For more fine-grained control of the traced execution paths, the language interpreter employs a set of annotations and hints to guide the meta-tracer as it executes the user program. Two of the most crucial annotations are \emph{jit\_merge\_point}, marking stable loop headers, and \emph{can\_enter\_jit}, indicating potential backward jumps. The logical program counter used for detecting loops is built from what are known as \emph{green} variables, which remain constant across multiple iterations, while other interpreter state variables that may change frequently are classified as \emph{red} variables. Both sets of variables, along with these hints, are registered with a \emph{JitDriver}, an object acting as a mediator between the interpreter and the global meta-tracer. An interpreter may define multiple JitDrivers, each with its own logical program counter, although there remains a single unified meta-tracer. We discuss this interplay further in \secref{section:hot-branches}, when addressing advanced trace optimizations.

        \paragraph{}%
            In this dissertation, we will study a concrete instance of a language interpreter built using meta-tracing: Pycket, an implementation of the Racket language built on the RPython framework. We will introduce Pycket in detail in the following section. Due to the use of meta-tracing techniques, Pycket’s tracing JIT compiler specifically captures and optimizes hot loops found within the user-level Racket code, rather than within its own interpreter logic. To understand this clearly, we will now examine concretely how an interpreter communicates loop information to a meta-tracer, as illustrated by \figref{fig:pycket-annotated-cek}.

        \paragraph{}%
            In this dissertation, we focus primarily on Pycket, an interpreter for the Racket language initially developed as a rudimentary implementation using RPython’s meta-tracing framework. Throughout the study, we progressively transform Pycket from this initial interpreter into a full-scale implementation of Racket, specifically by self-hosting Racket on top of Pycket itself. This evolution is facilitated by the meta-tracing approach, as Pycket's tracing JIT effectively identifies and optimizes hot loops in the user-level Racket code, rather than within the CEK interpreter logic. To understand concretely how such loop detection and optimization occurs, we next illustrate how the interpreter informs the meta-tracer about loops, as depicted in \figref{fig:pycket-annotated-cek}, as an example usage of meta-tracing hints.

        \inputSub{rpython}{pycket-annotated-cek}

        \paragraph{}%
            Concretely, the Pycket interpreter defines a dedicated \texttt{JitDriver}, explicitly specifying green and red variables to inform the meta-tracer about user-level loops. In Pycket’s CEK machine, the green variables consist primarily of the AST node currently being evaluated and a \texttt{came\_from} indicator, which together help encode recursive calls—the only way to make loops. Pycket’s interpreter further employs annotations such as \texttt{jit\_merge\_point} and \texttt{can\_enter\_jit} to explicitly mark stable loop entry points and potential backward jumps, respectively. These details are shown in \figref{fig:pycket-annotated-cek}, and we will further examine their operation in detail in the following section (\secref{section:pycket-primer}).

        \paragraph{}%
            Having described Pycket’s use of meta-tracing annotations, we now transition to exploring its internal core architecture. Specifically, we will discuss how its interpreter is built around the CEK abstract machine, a foundational design component that remains constant throughout our study—even as we modify and extend Pycket’s broader architecture during the transition to a self-hosted Racket implementation.

    \section[\texorpdfstring{Pycket: A Rudimentary Racket Interpreter Built on RPython Framework}{Pycket Primer}]{Pycket: A Rudimentary Racket Interpreter Built on RPython Framework}
        \label{section:pycket-primer}

        \begin{paragraph-here}
            Pycket, first designed as a high-performance tracing JIT compiler for Racket, supporting wide variety of complex language features such as contracts, continuations, structures, dynamic binding, and more. It's shown to outperform existing compilers at the time (2015) including Racket's own JIT \cite{pycketmain}, later shown to eliminate approximately \%90 of the sound gradual typing overhead \cite{pycketmain2}. In this section we provide a primer on Pycket's fundamentals. It's language RPython, CEK core, and how it is built, etc.
        \end{paragraph-here}

        \begin{paragraph-here}
            What's RPython?

            Originally created for PyPy.
            RPython (Restricted Python) is a statically typed, object-oriented
            proper subset of Python that is designed during PyPy's development
            \cite{pypy06}. It restricts Python in a way that enables
            type-inference on RPython programs allowing an efficient conversion to
            a lower language such as C. The restrictions are mainly the dynamic
            features of Python and commonly listed as;
            no mix types at the same location (RPython is a statically typed language, despite that it doesn't have explicit type annotations. This rule says that all variables need to be type
            consistent and infereable), run-time reflection is not supported
            (i.e. changing method of classes at run-time), no closures, global and
            class bindings are assumed to be constants etc \cite{rpython07,rpython09}.
        \end{paragraph-here}

        \inputSub{rpython}{rpython-toolchain-diagram}

        \begin{paragraph-here}
            What's RPython framework? What does it do? How does it do it? It generates JIT compilers from given interpreters.

            The RPython framework takes a language interpreter code written in RPython and translates it to a chosen
            lower-level language, most commonly C. Given an interpreter written in
            RPython, the toolchain translates it to the target language by
            lowering it in numerous phases, as can be seen in \figref{fig:rpython-toolchain}, where each phase has its own type
            system and a generic type inference engine. Because RPython is a
            subset of Python, the entire process can use a Python run-time (this also means that any Python interpreter can run that code). The
            general translation process can be described as follows. It starts
            with loading the RPython program into the run-time and get the function
            objects in memory as inputs. Using these function objects, the
            framework generates by abstract interpretation the control flow graphs
            for these functions that will be processed at further transformation
            steps. Then the annotator acts as a high-level type inference engine
            and assigns ``annotations'' to each variable at each flow graph. These
            annotations basically denotes the possible Python objects a variable
            might contain at the run-time. After the whole program is annotated,
            RTyper (RPython typer) takes control and starts lowering the
            annotations and some operations into the lower types and operations
            that would make sense to the targeted environment (e.g. structs,
            pointers, arrays in case of C), acting as a bridge between the
            high-level annotator and the low-level code generation. After RTyping,
            some optional backend optimizations are applied, such as inlining,
            malloc removal and escape analysis. Note that like Python, RPython is
            garbage collected, however, C is not. Therefore at this point a
            garbage collector is inserted into the program as well. Finally the
            typed and lowered flow-graphs are inputted into the code generation
            and the binary is generated. \cite{rpython07, pypy06, pypy08}
        \end{paragraph-here}

        \begin{paragraph-here}
            Enter Pycket, generated by that RPython framework.

            Pycket is first designed in 2014 as a high-performance JIT compiler
            for Racket, generated using the RPython meta-tracing framework
            \cite{bolzMetatracingMakesFast2014}. In its original design, as shown in
            \figref{fig:old-pycket}, Pycket relies on the Racket executable to
            read and fully expand a given module\cite{samth:11}, and then
            generates RPython AST for it and evaluates it within the interpreter
            loop \cite{pycketmain}. The language interpreter is based on the CEK
            abstract machine and has the state $\langle e, \rho, \kappa \rangle$ ($e$ : control
            (program AST), $\rho$ : environment, $\kappa$ : continuation)
            \cite{felleisen87}.
        \end{paragraph-here}

        \inputSub{rpython}{old-pycket-runtime-flow-diagram}

        \begin{paragraph-here}
            Macros \& Modules

            Pycket uses Racket’s macro expander \cite{flatt:2002} to evaluate macros, thereby reducing Racket programs to just a few core forms implemented by the runtime system \cite{samth:11}.
            To run a Racket program,2 Pycket uses Racket to macro-expand all the modules used in a program and write the resulting forms and metadata as JSON encoded files. Pycket then reads the serialized representation, parses it to an AST, and executes it. Adopting this technique enables Pycket to handle most of the Racket language while focusing on the key research contributions. Because Racket’s static optimizations are performed after macro expansion, Pycket does not benefit from them.
        \end{paragraph-here}

        \begin{paragraph-here}
            Assignment Conversion and ANF

            Once a module is expanded to core Racket and parsed from the serialized representation, Pycket performs two transformations on the AST. First, the program is converted to A-normal form (ANF), ensuring that all non-trivial expressions are named (\cite{danvy:93,flanagan:93}.

            Next, we convert all mutable variables (those that are the target of set!) into heap-allocated cells. This is a common technique in Lisp systems, and Racket performs it as well. This approach allows environments to be immutable mappings from variables to values. Additionally, each AST node stores its static environment.

        \end{paragraph-here}

        \begin{paragraph-here}
            Primitives and Values

            Racket comes with over 2,000 primitive functions and values, of which Pycket implements nearly 1500. These range from numeric operations, where Pycket implements the full numeric tower including bignums, rational numbers, and complex numbers, to regular expression matching, to input/output including a port abstraction. As of this writing, more than half of the non-test lines of code in Pycket implement primitive functions. These primitives are extended with the set of primitives we get from the bootstrapping linklets from Racket language, as we'll discuss in \chapterRef{chapter:linklets}.
        \end{paragraph-here}

        \begin{paragraph-here}
            hidden classes

            Pycket adapts the technique of hidden classes from prototype-based object systems to develop a novel implementation of proxies, a common mechanism used to enforce soundness in gradual type systems. Our approach not only improves the performance of gradually typed programs, but also other programs that use proxies created by Racket’s contract systems \cite{findler-felleisen:2002, pycketmain2}.
            Pycket also benefits from type specialization that naturally comes with tracing, as opposed to method-based JITs needing a separate "type-feedback" pass to decide which specializations to generate \cite{typeSpecial:2009}.
        \end{paragraph-here}

        \begin{figure}[t]
        \centering
        %--------------------------------------------------------------------
        % Syntax of expressions and continuations
        \[
        \begin{array}{lcl}
            e &::=& x \mid \lambda x.\,e \mid e\,e \\[4pt]
            \kappa &::=& [\,]
                    \mid \mathsf{arg}(e,\rho)::\kappa
                    \mid \mathsf{fun}(v,\rho)::\kappa
        \end{array}
        \]

        %--------------------------------------------------------------------
        % Transition rules of the CEK machine
        \[
        \begin{array}{rcl}
            \langle x,\rho,\kappa \rangle
            &\longmapsto&
            \langle \rho(x),\rho,\kappa \rangle \\[6pt]

            \langle (e_1\,e_2),\rho,\kappa \rangle
            &\longmapsto&
            \langle e_1,\rho,\mathsf{arg}(e_2,\rho)::\kappa \rangle \\[6pt]

            \langle v,\rho,\mathsf{arg}(e,\rho')::\kappa \rangle
            &\longmapsto&
            \langle e,\rho',\mathsf{fun}(v,\rho)::\kappa \rangle \\[6pt]

            \langle v,\rho,\mathsf{fun}(\lambda x.\,e,\rho')::\kappa \rangle
            &\longmapsto&
            \langle e,\rho'[x \mapsto v],\kappa \rangle
        \end{array}
        \]
        \caption{The CEK machine for the $\lambda$‑calculus \cite{pycketmain}.}
        \label{fig:cek-formal}
        \end{figure}

        \begin{paragraph-here}

            With the user program in its final form, it's ready to be evaluated using the CEK machine.
            To review, the CEK machine is described by the four transition rules in Figure~\ref{fig:cek-formal}.
            A CEK state has the form $\langle e,\rho,\kappa\rangle$ where $e$ is the AST for the program (the control), $\rho$ is the environment (a mapping of variables to values), and $\kappa$ is the continuation.
            A continuation is a sequence of frames and there are two kinds of frames:
            $\mathsf{arg}(e,\rho)$ represents the argument of a function application that is waiting to be evaluated, and
            $\mathsf{fun}(v,\rho)$ represents a function that is waiting for its argument.
            The first transition evaluates a variable by looking it up in the environment.
            The second and third transitions concern function applications; they reduce the function and argument expressions to values.
            The fourth transition performs the actual function call.
            Because no continuations are created when entering a function, tail calls are space efficient.
        \end{paragraph-here}

        \begin{paragraph-here}
            The RPython code in \figref{fig:pycket-annotated-cek} continuously transforms a CEK triple (ast, env, cont) into a new one, by calling the interpret method of the ast, with the current environment env and continuation cont as an argument. This process goes on until the continuation is the empty continuation, in which case a Done exception is raised, which stores the return value.
        \end{paragraph-here}


        \begin{paragraph-here}
               Pycket identifies loops using two complementary techniques: the two-state representation and a call graph approach. In the two-state representation, Pycket tracks each invocation of a loop or function call site by pairing the current and previous abstract syntax tree (AST) nodes, allowing loops to be recognized even if they start mid-way through the iteration (a "phase shift"). This mechanism treats recursive invocations separately by distinguishing recursive call sites, enabling Pycket to precisely generate specialized traces for each recursive loop site rather than only one generic trace per function. However, when recursive call patterns become more complex (such as indirect recursion or calls via proxies), the simple AST-node-based identification may fail. To address this, Pycket supplements loop detection with a dynamic call graph technique: it incrementally builds a runtime call graph representing invoked functions as nodes and calls as edges. Each edge is checked at runtime to detect cycles, indicating loops, even through proxies or indirect calls. This combined approach effectively identifies loops both at the straightforward AST level and through more complex runtime patterns, ensuring broader coverage of loop detection and generating optimized traces\cite{pycketmain}.
        \end{paragraph-here}


        \begin{paragraph-here}
            We'll talk about this Pycket's overall performance characteristics in \secref{section:pycket-performance-characteristics}. let's first talk about relevant RPython trace optimizations to provide a context for technical discussions about performance in \chapterRef{chapter:problem}.
        \end{paragraph-here}


    \section{Trace Optimizations \& Runtime Feedback}

    \begin{paragraph-here}
        Traces are the real performance currency of a tracing JIT compiler.

        Good trace quality is a thing.
    \end{paragraph-here}


RPython’s trace optimizer includes a suite of standard compiler optimizations, such as commonsubexpression elimination, copy propagation, constant folding, and many others (Ardo l. 2012). One advantage of trace compilation for optimization is that the control-flow graph of a trace is a straight line. Trace optimizations and their supporting analyses can be implemented in two passes over the trace, one forward pass and one backward pass.

Inlining Inlining is a vital compiler optimization for high-level languages, both functional and object-oriented. In a tracing JIT compiler such as RPython, inlining comes for free from tracing (Gal et al. 2006). A given trace will include the inlined code from any functions called during tracing. This includes Racket-level functions as well as runtime system functions (Bolz et al. 2009). The highlyaggressive inlining produced by tracing is one of the keys to its successful performance: it eliminates function call overhead and exposes opportunities for other optimizations.

Loop-invariant Code Motion Loop-invariant code motion is implemented in RPython particularly simple way, by peeling off a single iteration of the loop, and then performing its standard suite of forward analyses to optimize the loop further (Ardo 2012).

Allocation Removal The CEK machine allocates a vast quantity of objects which would appear in the heap without optimization. This ranges from the tuple holding the three components of the machine state, to the environments holding each variable, to the continuations created for each operation.

        \begin{paragraph-here}
            warmup is a thing
        \end{paragraph-here}

        \begin{paragraph-here}
            introduce and explain promote, give an example from Pycket

            promote, turns arbitrary variables into constants
        \end{paragraph-here}

        \begin{paragraph-here}
            talk about escape analysis, and introduce and explain trace-elidable, give an example from pycket

            @elidable, it's like memoization of functions, for traces.

            a function is trace-elidable, if during the execution of the program, successive calls to the function with identical arguments always return the same result.
        \end{paragraph-here}
