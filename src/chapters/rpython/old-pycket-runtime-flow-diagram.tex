\begin{figure}[!h]
  \centering
  \hspace*{-32mm}
  \begin{tikzpicture}[
      font=\footnotesize,
      box/.style  ={draw, rounded corners, align=center,
                    minimum width=32mm, minimum height=8mm, fill=gray!8},
      circ/.style ={draw, circle, minimum size=10mm, inner sep=1pt, fill=gray!8},
      spine/.style={-Stealth, thick},
      carr/.style ={-Stealth, thick},
      ann/.style  ={anchor=east, inner sep=0pt, font=\footnotesize},
      node distance=10mm
    ]

    % vertical nodes ----------------------------------------------------
    \node[circ] (n1) {input.rkt};
    \node[box,  below=of n1] (n2) {Pycket\\Entry};
    \node[box,  below=of n2] (n3) {Racket\\Executable};
    \node[box,  below=of n3] (n4) {Pycket\\Expand};
    \node[box,  below=of n4] (n5) {JITted CEK};

    % spine arrows ------------------------------------------------------
    \foreach \a/\b in {n1/n2,n2/n3,n3/n4,n4/n5}
      {\draw[spine] (\a.south) -- (\b.north);}

    % helper for curved annotation arrows -------------------------------
    \newcommand{\carrow}[3]{%
      \draw[carr] (#1.south west) to[out=180,in=180,looseness=1.25] (#2.north west);
      \node[ann] at ($(#1.west)+(-6mm,-10mm)$) {#3};
    }

    % curved arrows with labels -----------------------------------------
    \carrow{n2}{n3}{input.rkt}
    \carrow{n3}{n4}{input.rkt (expanded)}
    \carrow{n4}{n5}{Pycket AST}

  \end{tikzpicture}
  \caption{Pycket's original design used to run the Racket executable to expand before running it on the CEK.}
  \label{fig:old-pycket-workflow-vertical}
\end{figure}
