\begin{figure}[!htbp]                    % ordinary floating figure
  \centering
  \begin{minipage}{0.5\textwidth}
    \begin{lstlisting}[style=rptrace-style]
# start of the trace (preamble)
label(p0, p1)                                         1
guard_not_invalidated()                               2
guard_class(p0, ConsEnv)                              3
p3 = getfield_gc_r(p0, ConsEnv.prev)                  4
guard_class(p3, ConsEnv)                              5
i5 = getfield_gc_i(p3, Fixnum)                        6
i6 = getfield_gc_i(p0, Fixnum)                        7
i7 = int_add_ovf(i5, i6)                              8
guard_no_overflow()                                   9
guard_class(p1, LetCont)                              10
p9 = getfield_gc_r(p1, LetCont.ast)                   11
guard_value(p9, ConstPtr(ptr10))                      12
p11 = getfield_gc_r(p1, LetCont.env)                  13
p12 = getfield_gc_r(p1, LetCont.prev)                 14

# peeled-iteration (inner loop)
label(p11, i7, p12, "64723392")                       15
guard_not_invalidated()                               16
guard_class(p11, ConsEnv)                             17
i14 = getfield_gc_i(p11, Fixnum)                      18
i15 = int_add_ovf(i14, i7)                            19
guard_no_overflow()                                   20
guard_class(p12, LetCont)                             21
p17 = getfield_gc_r(p12, LetCont.ast)                 22
guard_value(p17, ConstPtr(ptr18))                     23
p19 = getfield_gc_r(p12, LetCont.env)                 24
p20 = getfield_gc_r(p12, LetCont.prev)                25
jump(p19, i15, p20, "64723392")                       26
\end{lstlisting}
  \end{minipage}
  \caption{A trace in a tracing JIT compiler is a linear sequence of machine code instructions.}
  \label{fig:trace}
\end{figure}
