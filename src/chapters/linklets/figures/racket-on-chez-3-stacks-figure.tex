% % --- in any .tex file that already loads TikZ -------------------------------
% \definecolor{racketclr}  {gray}{0.92}  % Racket
% \definecolor{schemeclr}  {gray}{0.78}  % Scheme
% \definecolor{cclr}       {gray}{0.55}  % C

% \begin{figure}[!htbp]
%   \centering
%   \begin{tikzpicture}[x=1cm,y=1cm]

%     % helper macro ----------------------------------------------------------------
%     \newcommand{\drawbar}[4]{%
%       \begin{scope}[shift={(#1,0)}]
%         \def\w{#2}\def\cum{0}%
%         \foreach \clr/\lab/\h in {#3}{%
%           \fill[\clr,draw=black] (0,\cum) rectangle (\w,\cum+\h);
%           \node[font=\small] at (\w/2,\cum+0.5*\h) {\lab};
%           \pgfmathsetmacro\cum{\cum+\h}%
%         }
%         \node[font=\scriptsize,above] at (\w/2,\cum) {#4};
%       \end{scope}}
%     %---------------------------------------------------------------------------

%     % bar 1: Racket v7
%     \drawbar{0}{2.2}{
%       racketclr/Racket expander/1.50,
%       schemeclr/regexp/0.30,
%       schemeclr/I\slash O/0.75,
%       cclr/rktio/0.70,
%       schemeclr/threads/0.75,
%       schemeclr/control+structs/0.55,
%       schemeclr/builtins/5.20,
%       schemeclr/bytecode+JIT compiler/1.15,
%       schemeclr/gc/0.55}{231k}
%     \node[font=\bfseries] at (1.1,-0.6) {Racket v7};

%     % bar 2: Chez Scheme
%     \drawbar{4}{2.2}{
%       schemeclr/Scheme expander/0.40,
%       schemeclr/builtins/2.95,
%       schemeclr/machine-code compiler/1.50,
%       schemeclr/kernel/0.80,
%       schemeclr/gc/0.10}{115k}
%     \node[font=\bfseries] at (5.1,-0.6) {Chez Scheme};

%     % bar 3: Racket-on-Chez Scheme
%     \drawbar{8}{2.2}{
%       racketclr/main/0.05,
%       racketclr/Racket expander/1.50,
%       racketclr/linklet/0.10,
%       racketclr/schemify/0.20,
%       schemeclr/I\slash O/0.80,
%       cclr/rktio/0.70,
%       schemeclr/threads/0.35,
%       schemeclr/control+structs+\ldots/0.75,
%       schemeclr/Scheme expander/0.40,
%       schemeclr/builtins/2.95,
%       schemeclr/machine-code compiler/1.50,
%       schemeclr/kernel/0.80,
%       schemeclr/gc/0.10}{208k}
%     \node[font=\bfseries] at (9.1,-0.6) {Racket on Chez Scheme};

%     % legend --------------------------------------------------------------------
%     \begin{scope}[shift={(13,3)}]
%       \draw[black,fill=racketclr] (0,0) rectangle (0.4,0.4);
%       \node[right] at (0.5,0.2) {Racket};
%       \draw[black,fill=schemeclr] (0,-0.7) rectangle (0.4,-0.3);
%       \node[right] at (0.5,-0.5) {Scheme};
%       \draw[black,fill=cclr] (0,-1.4) rectangle (0.4,-1.0);
%       \node[right] at (0.5,-1.2) {C};
%     \end{scope}

%   \end{tikzpicture}
%   \caption{Component sizes of three Racket implementations (dummy caption).}
%   \label{fig:racket-stacks}
% \end{figure}

\begin{figure}[!htbp]                    % ordinary floating figure
    \centering
    \includegraphics[width=0.9\linewidth]{\inputFigPath{linklets}{racket-portable.png}}
    \caption{Comparison of Racket implementations. Figure used from \cite{icfp2019}}
    \label{fig:racket-on-chez}
\end{figure}