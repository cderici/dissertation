\begin{figure}[htbp]
  \centering
  \begin{tikzpicture}[
    font=\footnotesize,
    instancebox/.style={draw, rounded corners=2pt, minimum width=28mm, minimum height=9mm, fill=white, align=center},
    linkletbox/.style={draw, rounded corners=2mm, minimum width=49mm, minimum height=22mm, fill=gray!16, font=\ttfamily\scriptsize, inner sep=2mm, text width=44mm, align=left},
    arr/.style={-Stealth, thick}
  ]
    % Input instances
    \node[instancebox] (a) at (0, 0.7) {Instance\\\texttt{a~$\leftarrow$~5}};
    \node[instancebox, below=8mm of a] (b) {Instance\\\texttt{b~$\leftarrow$~8}};

    % Curly brace (right)
    \draw [decorate,decoration={brace,amplitude=8pt,mirror}]
      ([xshift=4mm]b.south east) -- ([xshift=4mm]a.north east);

    % Get center of brace for arrow start
    \path ([xshift=15mm]a.north east) -- coordinate (bracecenter) ([xshift=5mm]b.south east);

    % Main linklet box
    \node[linkletbox, right=15mm of bracecenter, anchor=west] (lk) {
      \texttt{(linklet\\\ \ ((a) (b))\\\ \ (c)\\\ \ (define-values (c) (+ a b))\\\ \ (set! c 42))}
    };

    % Arrow from brace to linklet
    \draw[arr] (bracecenter) -- (lk.west);

    % Output instance
    \node[instancebox, right=10mm of lk.east, anchor=west] (c) {Instance\\\texttt{c~$\leftarrow$~42}};

    % Arrow: linklet to output instance
    \draw[arr] (lk.east) -- (c.west);

  \end{tikzpicture}
  \caption{Regular Instantiation of a Linklet}
  \label{fig:regular-linklet-instantiation}
\end{figure}
