\chapter[\texorpdfstring{PROGRAMMING LANGUAGES \& PORTABILITY}
                          {Linklets}]{PROGRAMMING LANGUAGES \& PORTABILITY}
\label{chapter:linklets}

	\begin{chaptersynopsis}

		Making programming languages more portable is good, and here's a way to do that.

	\end{chaptersynopsis}

	\begin{todo}[Import]
			Section 3.3.1 On Racket's Portability of the proposal document is relevant for this section.
		\end{todo}

% This is not only about Racket, the CPython is written in C. Python developers don't want to write C, they
% want to write Python....

% ICFP'19 experience report

	\section{Enriching Compiler \& Runtime Communication}

		\begin{mainpoint}
			Making programming languages more portable is good.

			It requires improving communication between the compiler and runtime.
		\end{mainpoint}

		\begin{todo}[Import]
			Section 3.2 of the proposal document is relevant for this section.
		\end{todo}

	\section{Linklets as Compilation Units}
		\label{section:linklet-semantics}

		\begin{mainpoint}
			Linklets are good tools to improve communication between the compiler and runtime.
		\end{mainpoint}

		\begin{todo}[Import]
			Section 3.2.1 Top-level REPL via Linklets of the proposal document is relevant for this section.
		\end{todo}

		\begin{todo}[Import]
			Section 3.2.2 Operational Semantics of Linklets of the proposal document is relevant for this section.

			Read it before reading the quals document.
		\end{todo}

		\begin{todo}[Import]
			Use sections from the Quals written document for linklets redex model, standard reduction relation, instantiation semantics, etc.
		\end{todo}

		\subsection{Operational Semantics of Linklets}

			\begin{mainpoint}
				Linklets have well-defined operational semantics.
			\end{mainpoint}

		\subsection{Executable Redex Model for Linklets}

			\begin{mainpoint}
				Operational semantics of linklets do work.
			\end{mainpoint}


	% This is a figure in landscape orientation
	% \begin{sidewaysfigure}
	% 		\includegraphics[width=\textwidth]{\figPath{pl-portability}/exampleFigure.png}
	% \caption{This is another example Figure, rotated to landscape orientation.}
	% \label{LandscapeFigure}
	% \end{sidewaysfigure}
