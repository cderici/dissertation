\chapter{Performance in Tracing and Meta-Tracing JIT Compilation}

    \begin{chapterpoint}
        Traces are the real performance currency of a tracing JIT compiler.

        Understanding performance-related concepts, such as traces, warmup, and relevant RPython backend optimizations, as well as reviewing Pycket's characteristics is important for further content in this part.
    \end{chapterpoint}

    \section{Meta-tracing in JIT Compilation}
        \begin{mainpoint}
            The problem that PyPy solves:

            Rather than the loops in the interpreter being evaluated, meta-tracing manages to capture the hot loops in the user program being interpreted.
        \end{mainpoint}

    \section{Pycket's Performance Characteristics}
        \begin{mainpoint}
            We introduced in earlier Pycket studies the performance characteristics of Pycket, and already foresaw a part of the problem with self-hosting we're going to discuss in the rest of this part.
        \end{mainpoint}


