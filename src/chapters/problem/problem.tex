\chapter{Fundamental Problems with Self-Hosting on a Tracing JIT}

	\label{chapter:problem}

	\begin{chapterpoint}
		There are two critical problems fundamental to self-hosting on a meta-tracing JIT:
		\begin{itemize}
			\item The data-dependent interpreter-style code leads to overspecialized, non-reusable traces.
			\item Computations like program expansion that's critical to self-hosting is not suitable for meta-tracing (because it's all static computation, there's no loop).
		\end{itemize}
		There are a bunch of performance issues spawn from these two problems.
	\end{chapterpoint}

	\begin{todo}[Import]
			Section 3.4 Performance Issues with Self-Hosting on JIT Compilers of the proposal document is relevant for this section.
	\end{todo}


	\section{Tracing Branch-Heavy Code}
		\begin{mainpoint}
		 	Tracing JITs suck at non-tight loops more than they thrive at tight ones.

			Branchiness of a source code by itself doesn't make sense. We need a computation for talking about branchiness.

			The data-dependent interpreter-style code leads to overspecialized, non-reusable traces.
		\end{mainpoint}

	\section{The Nature of the Beast: Tracing Data-Dependent Code}
		\begin{mainpoint}
			Self-hosting is a different problem from the one meta-tracing solves.

			There is no loop.

			This problem requires a significant amount of research to solve.
		\end{mainpoint}



		\inputSub{problem}{reg-match-figure}

		\paragraph{}%
		Tracing interpreter-style programs with complex control-flow paths is
		a known weakness of JIT compilers. The large number of indirections
		not only cripple the JIT optimizations but also causes the loops to be
		segmented into many highly data driven traces. For
		example, consider the following program in \figref{fig:regexp}, which
		is a very simple regular expression matcher. It is highly simplified
		and some of the rules are removed for space.

		%
		\paragraph{}%
		Tracing this program running with an input regexp, say
		$\mathtt{\#}$\racketcode{rx"defg"}, trying to match it against an
		input string produces a trace that follows the control-flow path of
		the program for that input, making tracing quite wasteful because for
		any other input regexp, say $\mathtt{\#}$\racketcode{rx"a*"} which
		follows an entirely different path on the program, the JIT produces,
		compiles and optimizes yet another trace for that input, unable to use
		the previously generated trace. This problem not only increases the
		warmup time but also produces traces that are unlikely to be
		frequently re-used.