\chapter{Fundamental Problems with Self-Hosting on a Tracing JIT}

	\label{chapter:problem}

	\begin{chapterpoint}
		There are two critical problems fundamental to self-hosting on a meta-tracing JIT:
		\begin{itemize}
			\item The data-dependent interpreter-style code leads to overspecialized, non-reusable traces.
			\item Computations like program expansion that's critical to self-hosting is not suitable for meta-tracing (because it's all static computation, there's no loop).
		\end{itemize}
		There are a bunch of performance issues spawn from these two problems.
	\end{chapterpoint}

	\begin{todo}[Import]
			Section 3.4 Performance Issues with Self-Hosting on JIT Compilers of the proposal document is relevant for this section.
	\end{todo}


	\section{Tracing JIT vs Branch-Heavy Code}
		\begin{mainpoint}
		 	Tracing JITs suck at non-tight loops more than they thrive at tight ones.

			Branchiness of a source code by itself doesn't make sense. We need a computation for talking about branchiness.

			The data-dependent interpreter-style code leads to overspecialized, non-reusable traces.
		\end{mainpoint}

	\section{The Nature of the Beast}
		\begin{mainpoint}
			Self-hosting is a different problem from the one meta-tracing solves.

			There is no loop.

			This problem requires a significant amount of research to solve.
		\end{mainpoint}
