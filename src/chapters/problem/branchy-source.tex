\begin{figure}[tb]
\centering
\begin{tikzpicture}[remember picture,
every node/.style={font=\sffamily\scriptsize},
callout/.style={draw,circle,red!80!black,thick,
                inner sep=0pt,minimum size=13pt}]
%----------------------------------------------------
% 1 ▸ Code listing (unchanged, pure ASCII)
\node[anchor=north west] (code) at (0,0) {%
\begin{minipage}{\linewidth}
\begin{lstlisting}[language=racket,
                basicstyle=\ttfamily\footnotesize,
                numbers=left, xleftmargin=2em]
#lang racket/base
(define (branchy-function lst)
    (letrec ([loop
            (lambda (lst)
            (if (null? lst)
                -1
                (let-values ([(input) (car lst)])
                    (if (> input 18)
                        (+ 18 (loop (cdr lst)))
                        (if (> input 8)
                            (+  8 (loop (cdr lst)))
                            (if (< input 3)
                                (if (< input 1)
                                    (+ 1 (loop (cdr lst)))
                                    (if (< input 2)
                                        (loop (cdr lst))
                                        (loop (cdr lst))))
                                (if (< input 5)
                                    (if (< input 4)
                                        (+ 3 (loop (cdr lst)))
                                        (+ 5 (loop (cdr lst))))
                                    (if (< input 6)
                                        (loop (cdr lst))
                                        (if (< input 7)
                                            (loop (cdr lst))
                                            (loop (cdr lst))))))))))])
    (loop lst)))
\end{lstlisting}
\end{minipage}};
%----------------------------------------------------
% 2 ▸ Set a local coordinate system: origin = top-left
\begin{scope}[
    shift={(code.north west)},
    x={(code.north east) - (code.north west)},   % horizontal
    y={(code.south west) - (code.north west)}]   % vertical (down)

% -- 2a  Branch-number circles (single column x = 0.97)
\coordinate (ColX) at (0.25,0.01);          % tweak this single number for the column
\node[callout] (n18) at ($(ColX)+(0,0.29)$) {18};  %   line  9  ≈ 0.28
\node[callout] (n8)  at ($(ColX)+(0.03,0.36)$) {8};   %   line 11  ≈ 0.38
\node[callout] (n3)  at ($(ColX)+(0.06,0.425)$) {3};   %   line 13  ≈ 0.47
\node[callout] (n1)  at ($(ColX)+(0.09,0.46)$) {1};   %   line 15  ≈ 0.56
\node[callout] (n5)  at ($(ColX)+(0.09,0.62)$) {5};   %   line 19  ≈ 0.69
\node[callout] (n4)  at ($(ColX)+(0.12,0.652)$) {4};   %   line 21  ≈ 0.78

% -- 2b  Exit-expression under-lines + arrows
\begin{scope}[exit/.style={green!60!black,very thick,->}]
% Exit A  (+ 18 …)  → line 9
\draw[exit] (0.58,0.335) -- node[midway,above=2pt]{Exit A} ++(0.25,0);
% Exit B  (+ 1 …)   → line 15
\draw[exit] (0.69,0.50) -- node[midway,above=2pt]{Exit B} ++(0.25,0);
% Exit C  (+ 3 …)   → line 21
\draw[exit] (0.73,0.695) -- node[midway,above=2pt]{Exit C} ++(0.25,0);
\end{scope}
\end{scope}
%----------------------------------------------------
\end{tikzpicture}
\caption{Branchy: a synthetic program designed to be branch-heavy and data-dependent.%
\ Numbers mark branch points; under-lined expressions are the three
exits from one iteration.}
\label{fig:branchy}
\end{figure}
