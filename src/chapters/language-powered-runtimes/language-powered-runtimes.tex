\chapter{Language-Powered Runtimes}
	\begin{chapterpoint}
		Languages can influence the capabilities of the runtime they are implemented on.

		What does a self-hosting language look like in practice?

	\end{chapterpoint}

	\section{Pycket as Platform for Self-Hosting Racket}

		\begin{mainpoint}
			Pycket is a suitable host and environment for proving the self-hosting hypothesis.
		\end{mainpoint}

		\begin{todo}[TODO]
			Introduce Pycket internals, characteristics

			RPython framework, etc.
		\end{todo}

	\section{Concrete Implementations of Linklets}

		\begin{mainpoint}
			Runtimes can represent linklets however they want.

			Chez Scheme represents them as functions, here's how Pycket represents and implements them.
		\end{mainpoint}

	\section{Enhancing Runtime with Bootstrapping Linklets}
		\begin{mainpoint}
			Importing functionality straight from the language in the form of linklets allows the language to shape the runtime.
		\end{mainpoint}
		\subsection{Interfacing with the Compiler: Expander Linklet}

		\subsection{Interfacing with the Host Environment: IO \& Thread Linklets}
			\begin{todo}[TODO]
				rktio
			\end{todo}
			\begin{todo}[TODO]
				engines
			\end{todo}
		\subsection{Language Utilities: Fasl \& Regexp Linklets}

