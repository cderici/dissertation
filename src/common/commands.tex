
\newcommand{\inputChapter}[1]{\input{chapters/#1/#1.tex}}
\newcommand{\inputSection}[1]{\input{sections/#1.tex}}
\newcommand{\inputMisc}[1]{\input{misc/#1.tex}}

\newcommand{\figPath}[1]{./chapters/#1/figures}

\newcommand{\inputSub}[2]{\input{chapters/#1/#2.tex}}

\newcommand\figref[1]{Figure~\ref{#1}}
\newcommand{\chapterRef}[1]{Chapter~\ref{#1}}
% Inline racket
\newcommand{\racketcode}[1]{\lstinline[basicstyle=\ttfamily\small,language=racket]{#1}}

%———————————————————————————————————————————————————————
%  PART ENTRIES IN THE TABLE OF CONTENT
%———————————————————————————————————————————————————————
\makeatletter
\renewcommand\thepart{\Roman{part}}

%   #1 = title text,   #2 = page number
\renewcommand*\l@part[2]{%
  \ifnum\c@tocdepth>-1\relax
    \addpenalty{-\@highpenalty}
    \addvspace{1.25em \@plus\p@}
    \begingroup
      \parindent \z@
      \rightskip \@pnumwidth
      \parfillskip -\rightskip
      \leavevmode
      \large\bfseries
      % actual text
      Part~\thepart\space#1\hfil\hb@xt@\@pnumwidth{\hss #2}%
      \par
      \addvspace{0.5em}
    \endgroup
  \fi}
\makeatother
