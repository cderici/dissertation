
\newcommand{\inputChapter}[1]{\input{chapters/#1/#1.tex}}
\newcommand{\inputSection}[1]{\input{sections/#1.tex}}
\newcommand{\inputMisc}[1]{\input{misc/#1.tex}}

% \inputFigPath{chapter-name}{figure-file-name}
\newcommand{\inputFigPath}[2]{chapters/#1/img/#2}

\newcommand{\figPath}[1]{./chapters/#1/figures}

\newcommand{\inputSub}[2]{\input{chapters/#1/#2.tex}}
\newcommand{\inputFigure}[2]{\input{chapters/#1/figures/#2.tex}}

\newcommand{\figref}[1]{Figure~\ref{#1}}
\newcommand{\chapterRef}[1]{Chapter~\ref{#1}}
\newcommand{\sectionRef}[1]{Section~\ref{#1}}
\newcommand\secref[1]{Section~\ref{#1}}
\newcommand\appendixRef[1]{Appendix~\ref{#1}}
\newcommand{\tableRef}[1]{Table~\ref{#1}}
% Inline racket
\newcommand{\racketcode}[1]{\lstinline[basicstyle=\ttfamily\small,language=racket]{#1}}
\newcommand{\pycketcode}[1]{\lstinline[basicstyle=\ttfamily\small,language=racket]{#1}}

\newcommand{\smartQL}{“}
\newcommand{\smartQR}{” }

\def\dash {\text{-}}
\def\la {\bm{L_\alpha}}
\def\lb {\bm{L_\beta}}

\newcommand{\confstk}[4]{\langle #1,\;#2,\;#3,\;#4\rangle}
\newcommand{\confcek}[4]{\langle #1,\;#2,\;#3,\;#4\rangle}
\newcommand{\Rule}[2]{\overset{\textsc{#1}}{\underset{\text{\tiny #2}}{\longrightarrow}}}
\newcommand{\sfm}[1]{\ifmmode\mathsf{#1}\else\textsf{#1}\fi}
\newcommand{\stk}{\sfm{stk}}
\newcommand{\cek}{\sfm{cek}}
\newcommand{\eps}{\varepsilon}      % empty continuation token


% reduction stuff
\newcommand{\redinput}[1]{EP\;\llbracket #1 \rrbracket, \rho, \sigma\;}
\newcommand{\redoutput}[3]{EP\;\llbracket #1 \rrbracket, #2, #3\;}

\newcommand{\rcinput}[1]{EP\;\llbracket E\;\llbracket #1 \rrbracket \rrbracket, \rho, \sigma\;}
\newcommand{\rcoutput}[3]{EP\;\llbracket E\;\llbracket #1 \rrbracket \rrbracket, #2, #3\;}
\def\rcrel {&\longrightarrow_{\beta_l}\;}

\def\where {\textbf{where}\;}
\def\rel {&\longrightarrow_{\beta_p}\;}

%———————————————————————————————————————————————————————
%  PART ENTRIES IN THE TABLE OF CONTENT
%———————————————————————————————————————————————————————
\makeatletter
\renewcommand\thepart{\Roman{part}}

%   #1 = title text,   #2 = page number
\renewcommand*\l@part[2]{%
  \ifnum\c@tocdepth>-1\relax
    \addpenalty{-\@highpenalty}
    \addvspace{1.25em \@plus\p@}
    \begingroup
      \parindent \z@
      \rightskip \@pnumwidth
      \parfillskip -\rightskip
      \leavevmode
      \large\bfseries
      % actual text
      Part~\thepart\space#1\hfil\hb@xt@\@pnumwidth{\hss #2}%
      \par
      \addvspace{0.5em}
    \endgroup
  \fi}
\makeatother
